\section{Case study: Flowcharts with Feedback}

\begin{enumerate}[start=3]
\item \textbf{As we happened to be working with \Fw\ to model the
  creative process, we will use \Fw\ as a case study to exemplify how
  to implement this writers workshop model to use feedback within
  being creative.}
\item[] What can we give feedback about in this context?
\begin{enumerate}
\item \emph{Population of nodes}: What can they do?  What do we learn when a
  new node is added?
\item \emph{Population of flowcharts}: Simon and Alison have talked
  about ``broken'' flowcharts; this suggests a sort of test-driven
  development framework.
\item \emph{Population of output texts}: How to generate commentary on
  a generated artefact?
\end{enumerate}
\item[] How will the feedback be understood and used?
\begin{enumerate}
\item Recombine nodes based on their input and output properties to
  assemble new flowcharts.  Potentially evaluate and evolve the
  population of nodes programmatically if we can define fitness
  functions.
\begin{itemize}
\item The role of temporality is interesting here: if the Workshop
  takes place in real time, this requires different approaches to
  composition that can take place more or less offline \cite{perez2013rolling}.
\end{itemize}
\item Possibly a new node will ``fix'' a broken flowchart.  We could also test existing flowcharts for worthwhile features (rewiring the flowcharts through cross-over, assessing the output at various stages)
\item Connect commentary on a generated artefact with the code that
  made that artefact.
\end{enumerate}
\end{enumerate}
