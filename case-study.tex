\section{Case study: Flowcharts and Feedback} \label{sec:implementation}

This section describes work that is currently underway to implement the
Writers Workshop model, not only within one system but as a new paradigm
for collaboration among disparate projects.  In order to
bring in other participants, we need a neutral environment that is not
hard to develop for: the \Fw\ system mentioned in Section
\ref{sec:ww-model} offers one such possibility.  The basic primary
objects in the \Fw\ system are \emph{flowcharts}, which are comprised of
interconnected \emph{process nodes}
\cite{charnley2014flowr,colton-flowcharting}.  Process nodes specify
input and output types, and internal processing can be implemented in
Java, or other languages that interoperate with the JVM, or by
invoking external web services.  One of the common applications to
date is to generate computer poetry, and we will focus on that domain
here.

A basic set of questions, relative to this system's components, are as 
follow:
\begin{enumerate}
\item \emph{Population of nodes}: What can they do?  What do we learn
  when a new node is added?
\item \emph{Population of flowcharts}: \citeA{pease2013discussion}
  have described the potentially-serendipitous repair of ``broken''
  flowcharts when new nodes become available; this suggests the need for
  test-driven development framework.
\item \emph{Population of output texts}: How to assess and comment on
  a generated poetic artefact?
\end{enumerate}

In a further evolution of the system, the sequence of steps in a
Writers Workshop could itself be spelled out as a flowchart.  The
process of reading a poem could be conceptualised as generating a
semantic graph  \cite{harrington2007asknet,francisco2006automated}.  Feedback could be modelled as
annotations to a text, including suggested edits.  These markup
directives could themselves be expressed as flowcharts.
A standardised set of markup structures may
partially obviate the need for strong natural language understanding, at
least in interagent communication.  Thus, we could agree that {\tt observations} will consist
of stand-off annotations that connect textual passages to public URIs using a limited
comparison vocabulary, and {\tt suggestions} will consist of simple
stand-off line-edits, which may themselves be marked up with rationale.  These
restrictions, and similar restrictions around constrained turn-taking,
could be progressively widened in future versions of the system.
%
The way the poems that are generated, the models of poems that are
created, and the way the feedback is generated, all depend on the
contributing system's body of code and prior experience, which may
vary widely between participating systems.  In the list of functional steps below, all of
the functions could have a subscripted ``$\mathcal{E}$'', which is
omitted throughout.  Exchanging path dependent points of view will tend to produce results
that are different from what the individual participating systems would
have come up with on their own.

\begin{enumerate}[label=\Roman*.]
\item Both the author and critic should be able to work with a model
  of the text.  Some of the text's features may be explicitly tagged
  as ``interesting.''  Outstanding questions may possibly be
  brought to the attention of critical listeners, e.g.~with the
  request to compare two different versions of the poem ({\tt
    presentation}, {\tt listening}).
\begin{enumerate}[label=\arabic*.]
\item \emph{A model of the text}. $m: T\rightarrow M$.
\item \emph{Tagging elements of interest}. $\mu: M\rightarrow I$.
\end{enumerate}
\item Drawing on its experience, the critic will use its model of the
  poem to formulate feedback ({\tt feedback}).
\begin{enumerate}[label=\arabic*.]
\item \emph{Generating feedback}. $f: (T,M,I)\rightarrow F$.
\end{enumerate}
\item Given the constrained framework for feedback, statements about
  the text will be straightforward to understand, but rationale for
  making these statements may be more involved ({\tt questions}, {\tt
    replies}).
\begin{enumerate}[label=\arabic*.]
\item \emph{Asking for more information}. $q: (M,F,I) \rightarrow Q$.
\item \emph{Generating rationale}. $a: (M,F,Q) \rightarrow \Delta F$.
\end{enumerate}
\item Finally, feedback may affect the author's model of the world, and the way future poems are generated ({\tt reflection}).
\begin{enumerate}[label=\arabic*.]
\item \emph{Updating point of view}. $\rho: (M,F) \rightarrow \Delta\mathcal{E}$.
\end{enumerate}
\end{enumerate}

The final step is perhaps the most interesting one, since it invites us
to consider how individual elements of feedback can ``snowball'' and
go beyond line-edits to a specific poem to much more fundamental
changes in the way the presenting agent writes poetry.  Here methods
for pattern mining, discussed in Section
\ref{ref:related-computational-serendipity}, are particularly relevant.
If systems can share code (as in our sample dialogue in Section
\ref{sec:dialogue-example}) this will help with the
rationale-generating step, and may also facilitate direct updates to
the codebase.  However, shared code may be more suitably placed into
the common pool of resources available to \Fw\ than copied over as
new ``intrinsic'' features of an agent.

Although different systems with different approaches and histories are
important for producing unexpected effects, ``offline'' programmatic
access to a shared pool of nodes and existing flowcharts may be
useful.  Outside of the workshop itself, agents may work to recombine
nodes based on their input and output properties to assemble new
flowcharts.  This can potentially help evaluate and evolve the
population of nodes programmatically, if we can use this sort of
feedback to define fitness functions.  The role of
temporality is interesting: if the workshop takes place in real time,
this will require different approaches to composition that takes place
offline \cite{perez2013rolling}.
%
Complementing these ``macro-level'' considerations, it is also worth
commenting on the potential role of ``micro-level'' feedback within
flowcharts.  Local evaluation of output from a predecessor node could
feed backwards through the flowchart, similar to backpropagation in
neural networks.  This would rely on a reduced version of the
functional schema described above.
