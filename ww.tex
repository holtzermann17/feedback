\documentclass[letter]{article}

\usepackage{ijcai15}
\usepackage{times}

\usepackage{eurosym}
\usepackage{hyperref}		% clickable references

\usepackage{apacite}            % APA style citations
\usepackage{apacdoc}

\usepackage{amsmath,amssymb}	% math structures and symbols
\usepackage{graphicx}		% including of graphics files in various formats
\usepackage{amsfonts}
\usepackage{dialogue}
\usepackage{placeins}
\usepackage{tikz}
\usetikzlibrary{positioning,backgrounds,fit,arrows,shapes,shadows}
\usetikzlibrary{shapes.multipart}
\usepackage{wrapfig}
\usepackage{enumitem}
\usepackage{framed}

% \hypersetup{hidelinks}

\usepackage[xspace,mla]{ellipsis}

\definecolor{myyellow}{RGB}{242,226,149}
\definecolor{mygreen}{RGB}{144,238,144}
\definecolor{mypink}{RGB}{255,182,193}
\definecolor{myorange}{RGB}{255,165,0}
\definecolor{myblue}{RGB}{0,204,204}

\usepackage{xparse}
\NewDocumentCommand\StickyNote{O{4cm}mmO{4cm}}{%
\begin{tikzpicture}
\node[
drop shadow={
  shadow xshift=2pt,
  shadow yshift=-4pt
},
inner xsep=7pt,
fill=#2,
xslant=-0.05,
yslant=0.05,
inner ysep=10pt
] {\parbox[t][#1][c]{#4}{#3}};
\end{tikzpicture}%
}

\urldef{\mailsa}\path|{j.corneli,s.colton}@gold.ac.uk|
\urldef{\mailsb}\path|a.pease@dundee.co.uk|

\newcommand{\Fw}{{\sf FloWr}}
\newcommand{\dec}[1]{\raisebox{.2ex}{$\star$}#1\raisebox{.2ex}{$\star$}}

\newcommand{\keywords}[1]{\par\addvspace\baselineskip
\noindent\keywordname\enspace\ignorespaces#1}

\begin{document}

% TITLE INFORMATION
\title{Implementing feedback in creative systems}

\author{Joseph Corneli\textsuperscript{1} and Anna Jordanous\textsuperscript{2}\\
\textsuperscript{1} Department of Computing, Goldsmiths College, University of London\\
\textsuperscript{2} School of Computing, University of Kent}

\date{today}

\maketitle

\begin{abstract} 
% First attempt
%% Various social strategies, ranging from Writers Workshops to open
%% source software, pair programming, and design charettes have been
%% developed to exploit emergent effects and to develop new shared
%% language.  We investigate the feasibility of using designs of this
%% sort in multi-agent systems that learn by sharing and discussing
%% partial understandings.  Building on earlier theoretical work, we
%% describe concrete implementation plans and some preliminary results.
\end{abstract}

\section*{\emph{Outline}} \label{sec:outline}

{\small
\begin{enumerate}[start=0,label=\textbf{\arabic*}]
\item \textbf{\emph{Implementing feedback in creative systems: a writers
  workshop approach.}}
\item[] Contributions:
\begin{enumerate}
\item \emph{Introduce} the writers workshop as a model for learning
  from feedback; and
\item \emph{Survey} the aspects of feedback that can be offered, and
  how they can be interpreted; and
\item \emph{Present} a case study on how to implement this, at a level
  that other people could use to implement similar Workshop designs in
  their own creative systems.
\end{enumerate}
\item \textbf{Here is the writers workshop idea.}
\item[] The steps:
\begin{enumerate}
\item A: {\tt presentation}
\item C: {\tt listening}
\item C: {\tt feedback} ({\tt observations}{\tt+}{\tt suggestions})
\item A: {\tt questions}
\item C: {\tt replies}
\item A: {\tt reflections}
\end{enumerate}
\item[] Previous applications:
\begin{enumerate}[label=(\roman*)]
\item We sketched a ``theory of poetics rooted in the making of
  boundary-crossing objects and processes''
\item We ``described a system that can (sometimes) make `highly
  serendipitous' creative advances in computer poetry'' while
  ``drawing attention to theoretical questions related to program
  design in an autonomous programming context.''
\end{enumerate}
\item \textbf{Here we refine the idea and turn it into a general model
  that for incorporating feedback within computational models of the
  creative process.}
\item[] What can feedback be about? \dec{Survey}
\begin{enumerate}
\item Patterns that \emph{match} and any \emph{exceptions}
\item Progress relative to explicit or adduced \emph{exploration}
  (knowledge and accuracy) or \emph{exploitation} (directional,
  task-based) goals.
\item \emph{Quantity}, \emph{Variety}, and \emph{Order} of produced
  objects or behaviours
\item New relationships among produced objects and behaviours drawing
  on a common field of reference
\end{enumerate}
\item[] How can feedback be understood and used? \dec{Survey}
\begin{enumerate}
\item Update knowledge base with new facts (accept statements,
  possibly with provenance)
\item Note similarity to \emph{iterative development}
\item Reflection: describe relationships between produced objects and
  behaviours and feedback
\item Reflection: Higher-order patterns e.g.~new patterns that
  describe the identified exceptions
\end{enumerate}
\item \textbf{As we happened to be working with \Fw\ to model the
  creative process, we will use \Fw\ as a case study to exemplify how
  to implement this writers workshop model to use feedback within
  being creative.}
\item[] What can we give feedback about in this context?
\begin{enumerate}
\item \emph{Population of nodes}: what can they do?  what do we learn when a
  new node is added?
\item \emph{Population of flowcharts}: Simon and Alison have talked
  about ``broken'' flowcharts; this suggests a sort of test-driven
  development framework.
\item \emph{Population of output texts}: how to generate commentary on
  a generated artefact?
\end{enumerate}
\item[] How will the feedback be understood and used?
\begin{enumerate}
\item ?
\item ?
\item Connect commentary on a generated artefact with the code that
  made that artefact.
\end{enumerate}
\item \textbf{Discussion of how this would work more generally in
  computational creativity and perhaps in AI more generally.}
\begin{enumerate}
\item Feedback is the fundamental concept in \emph{cybernetics}.  \dec{Definition?}
\item Feedback about feedback (\&c for higher orders) is relevant to thinking about \emph{learning} and \emph{communication}
\item \emph{Creativity} is often envisaged as cyclical process (e.g.~Dickie's
  art circle, Colton et al.~Iterative
  Development Expression Appreciation).  There are opportunities for
  embedded feedback at each step, and the process itself is ``akin
  to'' a feedback loop.
\end{enumerate}
\item \textbf{Conclusion: we have described a \emph{general} and \emph{computationally feasible} model for learning from feedback.}
\end{enumerate}
}

\include{oldtext}

\bibliographystyle{apacite}
\bibliography{./biblio}

\end{document}
