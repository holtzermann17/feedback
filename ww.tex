\documentclass[letter]{article}

\usepackage{ijcai15}
\usepackage{times}

\usepackage{eurosym}
\usepackage{hyperref}		% clickable references

\usepackage{apacite}            % APA style citations
\usepackage{apacdoc}

\usepackage{amsmath,amssymb}	% math structures and symbols
\usepackage{graphicx}		% including of graphics files in various formats
\usepackage{amsfonts}
\usepackage{dialogue}
\usepackage{placeins}
\usepackage{tikz}
\usetikzlibrary{positioning,backgrounds,fit,arrows,shapes,shadows}
\usetikzlibrary{shapes.multipart}
\usepackage{wrapfig}
\usepackage{enumitem}
\usepackage{framed}

% \hypersetup{hidelinks}

\usepackage[xspace,mla]{ellipsis}

\definecolor{myyellow}{RGB}{242,226,149}
\definecolor{mygreen}{RGB}{144,238,144}
\definecolor{mypink}{RGB}{255,182,193}
\definecolor{myorange}{RGB}{255,165,0}
\definecolor{myblue}{RGB}{0,204,204}

\usepackage{xparse}
\NewDocumentCommand\StickyNote{O{4cm}mmO{4cm}}{%
\begin{tikzpicture}
\node[
drop shadow={
  shadow xshift=2pt,
  shadow yshift=-4pt
},
inner xsep=7pt,
fill=#2,
xslant=-0.05,
yslant=0.05,
inner ysep=10pt
] {\parbox[t][#1][c]{#4}{#3}};
\end{tikzpicture}%
}

\urldef{\mailsa}\path|{j.corneli,s.colton}@gold.ac.uk|
\urldef{\mailsb}\path|a.pease@dundee.co.uk|

\newcommand{\Fw}{{\sf FloWr}}
\newcommand{\dec}[1]{\raisebox{.2ex}{$\star$}#1\raisebox{.2ex}{$\star$}}

\newcommand{\keywords}[1]{\par\addvspace\baselineskip
\noindent\keywordname\enspace\ignorespaces#1}

\begin{document}

% TITLE INFORMATION
\title{Implementing feedback in creative systems: A workshop approach}

\author{Joseph Corneli\textsuperscript{1} and Anna Jordanous\textsuperscript{2}\\
\textsuperscript{1} Department of Computing, Goldsmiths College, University of London\\
\textsuperscript{2} School of Computing, University of Kent}

\date{today}

\maketitle

\begin{abstract} 
% First attempt
%% Various social strategies, ranging from Writers Workshops to open
%% source software, pair programming, and design charettes have been
%% developed to exploit emergent effects and to develop new shared
%% language.  We investigate the feasibility of using designs of this
%% sort in multi-agent systems that learn by sharing and discussing
%% partial understandings.  Building on earlier theoretical work, we
%% describe concrete implementation plans and some preliminary results.
%%
%%
%%
\item[] Research question: \emph{How do we implement feedback in
  creative systems?}
%
\item[] Contributions:
\begin{enumerate}
\item \emph{Introduce} the writers workshop as a model for learning
  from feedback; and
%
\item \emph{Survey} the aspects of feedback that can be shared in the
  workshop setting, and how they can be interpreted; and
%
\item \emph{Present} a case study on how to implement this, at a level
  that other people could use to implement similar Workshop designs in
  their own creative systems.
%
\item \emph{Discuss} the broader relevance considering how our model
  relates to other aspects AI, communication, and culture.
  \emph{Serendipity} \emph{learning}\ldots
%

\end{enumerate}
\end{abstract}

\section{Introduction} \label{sec:introduction}

%AI and Feedback is the first international workshop on the topic and it is one of the IJCAI 2015 workshops. It focuses on applying AI techniques for addressing the challenges of mining and extracting feedback, as well as assessing, analysing, and making use of feedback.
%
%Feedback is key for both improvement and decision making. As humans, we are designed to constantly seek feedback on how and what we are doing in life. Feedback can come from ourselves, from our peers, from our teachers, from our collaborators, audiences, customers, public or press. Feedback provides opportunities to learn about how we and our work are perceived by others. If we encounter someone [something] new, we can examine previous feedback to learn how this new person [thing] is perceived by others.
%
%A key target of this workshop is to discuss how to build intelligent feedback agents that are capable of autonomously providing feedback that equals or surpasses that of human beings in its usefulness. The feedback of artificial feedback agents should have some desirable characteristics. It should be socially and culturally appropriate, clearly expressed, sufficiently focused and contextualised, thoughtfully challenging yet encouraging, compassionate, open to debate, justified and comparative, also, it should be trustworthy. Giving and receiving feedback with these characteristics therefore is a challenging, creative process.
%
%Aims and objectives. This workshop aims to bring together researchers from different strands of AI to discuss various aspects of feedback. The following questions are examples for consideration:

% * How can we build creative feedback agents that can generate feedback, especially on creative work?
% * How can we provide a sufficient agency in AI systems so that feedback from such systems will be useful?
% * How can we model feedback so that it can be generated by AI systems?
% * How can we design intuitive environments in which groups of humans and AI systems can give each other feedback?
% * How can we design systems that can act as creative collaborators? That not only give feedback, but can actually propose changes to a developing artefact?
% * How can we mine and learn from large datasets of feedback, e.g. educational social networks?
% * How can implicit feedback be extracted and interpreted?
% * How can trust and reputation models help with the assessment of feedback?
% * How can we process and use feedback?
% * What are the possible applications for AI systems capable of giving creative feedback?

%Topics of Interest
%
%Topics of interest cover a variety of feedback-related issues, such as mining and extracting feedback, generating feedback, understanding feedback, and assessing feedback.
%
%Topics of interest cover different strands of AI, such as multiagent systems, machine learning, natural language processing, and knowledge representation.
%
%Topics include, but are not limited to:
%
%Ontologies of feedback
%Multimodal feedback
%Implicit feedback
%Opinion mining and sentiment analysis
%Automatic generation of feedback
%Modelling the impact of feedback
%Designing environments for feedback
%Feedback and machine learning
%Trust and reputation models for feedback analysis
%Applications: creative industries, music composition, online learning, etc.

%We welcome and strongly encourage the submission of high quality, original work (published or unpublished), as well as visionary papers and roadmaps relevant to the scope of this workshop.
%
%Submitted papers must be formatted according to IJCAI guidelines. Formatting guidelines and electronic templates are available here.
%
%Submitted papers should not exceed 8 pages, excluding the bibliographic references.

In educational applications it would be useful to have an automated tutor that can read student work and make suggestions based on diagnostics, like, is the paper wrong, and if so how?  What background material should be recommended to the student for review?

In the current paper, we ``flip the script'' and look at what we believe to be a more fundamental problem for AI: computer programs that can themselves learn from feedback.  After all, if it was easy to build great automatic tutors, they would be a part of everyday life.  We look forward to a future when that is the case.

Computational creativity is a challenge within artificial intelligence where feedback plays a vital part [REFS***] \cite{perezyperez10MM,pease10}. Creativity cannot happen in a `silo' but instead is influenced and affected by feedback and interaction with others \cite{csik88,saunders2012towards}. Computational creativity researchers are starting to place more emphasis on social interaction and feedback by computational systems [***EVIDENCE ***]. Still, nearly 3 in 4 papers at the 2014 International Conference for Computational Creativity\footnote{ICCC is  the key international conference for research in computational creativity.} failed to acknowledge the role of feedback or social communication in their computational work on creativity. 

To highlight and contribute towards modelling feedback as a crucial part of creativity, we propose in this paper a model of computational feedback for creative systems based on Writers Workshops \cite{gabriel2002writer}, a literary collaborative practice that encourages interactive feedback within the creative process. We introduce Writers Workshops (Section \ref{sec:writers-workshop}), discuss their usefulness as a candidate for a computational model of feedback (Section \ref{sec:ww-analysis}) and propose such a model (Section \ref{sec:ww-model}). We partially evaluate our model by reflecting on how it fits with related work encouraging serendipity and emergence in computational models of intelligence and creativity \ref{sec:ww-related}. 
While we acknowledge that this paper is offering a roadmap for this model rather than a full implementation, we consider how the model could be practically implemented in a computational system and report our initial implementation work (Section \ref{sec:implementation}). We conclude by looking at ways in which this work can be usefully directed in the future.

%Current computer programs are able to identify patterns and ``close matches'' in data sets from certain domains, like music (David Meredith).  Learning \emph{new} patterns on the fly is harder but potentially quite useful. 
% AJ Don't see what this adds?


%Before we can learn from AI systems, we will have to teach them -- and learn how to learn together with them.  Accordingly, we \emph{design for emergence}.  (Say more.)
% AJ in this community, we need to be careful talking about emergence. It's been studied a lot and we'd need to reference a whole heap of work... Hence a suggested change in title below and above


%\input{oldtext}

\section{The Writers Workshop} \label{sec:writers-workshop}

Richard Gabriel \citeyear{gabriel2002writer} describes the practise of
Writers Workshops that has been put to use for over a decade within
the Pattern Languages of Programming (PLoP) community.  The basic
style of collaboration originated much earlier with groups of literary
authors who engage in peer-group critique.  Some literary workshops
are open as to genre, and happy to accommodate beginners, like the
Minneapolis Writers
Workshop\footnote{\url{http://mnwriters.org/how-the-game-works/}};
others are focused on professionals working within a specific genre,
like the Milford Writers
Workshop.\footnote{\url{http://www.milfordsf.co.uk/about.htm}}

The
practices that Gabriel describes are fairly typical:  
\begin{itemize}
\item Authors come with work ready to present, and read a short
  sample.
\item This work is generally work in progress (and workshopping is
  meant to help improve it).  Importantly, it can be early stage work.
  Rather than presenting a created artefact only, activities in the
  workshop can be aspects of the creative process itself.  Indeed, the
  model we present here is less concerned with after-the-fact
  assessment than it is with dealing with the formative feedback that
  is a necessary support for creative work.
\item The sample work is then
discussed and constructively critiqued by attendees.  Presenting
authors are not permitted to rebut these comments.  The commentators
generally summarise the work and say what they have gotten out of it,
discuss what worked well in the piece, and talk about how it could be
improved.  
\item The author listens and may take notes; at the end, he or
she can then ask questions for clarification.  
\item Generally, non-authors are either not permitted to attend, or
  are asked to stay silent through the workshop, and perhaps sit
  separately from the participating authors/reviewers.\footnote{Here
    we present Writers Workshops as they currently exist; however this
    last point is debatable. Whether non-authors should be able to
    participate or not is an interesting avenue for experimentation
    both in human and computational contexts.  The workshop dialogue
    itself may be considered an ``art form'' whose ``public'' may
    potentially wish to consume it in non-participatory ways.  Compare
    the classical Japanese \emph{renga} form \cite{jin1975art}.}
\end{itemize}

Essentially, the Writers Workshop is somewhat like an interactive peer review. The underlying concept is reminiscent of Bourdieu's {\em fields of cultural production} \cite{bourdieu93} where cultural value is attributed through interactions in a community of cultural producers active within that field. 

\subsection{Writers Workshop as a computational model}\label{sec:ww-model}

The use of Writers Workshop in computational contexts is not an
entirely new concept. In PLoP workshops, authors present design
patterns and pattern languages, or papers about patterns, rather than
more traditional literary forms like poems, stories, or chapters from
novels.  Papers must be workshopped at a PLoP or EuroPLoP conference
in order to be considered for the \emph{Transactions on Pattern
  Languages of Programming} journal.  A discussion of writers
workshops in the language of design patterns is presented by Coplien
and Woolf \citeyear{coplien1997pattern}. % Their patterns include
%\emph{Open Review}, \emph{Safe Setting}, \emph{Workshop Comprises
%  Authors}, \emph{Authors are Experts}, \emph{Community of Trust},
%\emph{Moderator Guides the Workshop}, \emph{Thank the Author},
%\emph{Selective Changes}, and \emph{Clearing the Palate}.
%
%A related reference that is particularly useful for readers who are
%not familiar with design pattern methods, \citeA{meszaros1998pattern}
%describe the typical scenario for authors of design patterns in a
%somewhat recursive manner, using a collection of design patterns,
%including \emph{Clear target audience}, \emph{Visible forces}, and
%\emph{Relationship to other patterns}.  \citeA{gamma1994design} is the
%best-know reference collecting design patterns for software.

The steps in the workshop can be distilled into the following phases,
each of which could be realised as a separate computational step in an
agent-based model:
\begin{center}
\begin{fminipage}{.53\columnwidth}
\begin{enumerate}[itemsep=0pt]
\item Author: {\tt presentation}
\item Critic: {\tt listening}
\item Critic: {\tt feedback}
\item Author: {\tt questions}
\item Critic: {\tt replies}
\item Author: {\tt reflections}
\end{enumerate}
\end{fminipage}
\end{center}

The {\tt feedback} step may be further decomposed into {\tt
  observations} and {\tt suggestions}.  This protocol is what we have
in mind in the following discussion of the Writers
Workshop.\footnote{The connections between Writers Workshops and
  design patterns, noted above, appear to be quite natural, in that
  the steps in the workshop protocol roughly parallel the typical
  components of design pattern templates: \emph{context},
  \emph{problem}, \emph{solution}, \emph{rationale}, \emph{resolution
    of forces}.}

\subsubsection{Dialogue example} \label{sec:dialogue-example}
Note that for the following dialogue to be possible computationally,
it would presumably have to be conducted within a lightweight process
language.  Nevertheless, for convenience, the discussion will be
presented here as if it was conducted in natural language.  Whether
contemporary systems have adequate natural language understanding to
have interesting interactions is one of the key unanswered questions
of this approach, but protocols such as the one described above are sufficient to make the experiment.

For example, here's what might happen in a discussion of the first few
lines of a poem, ``On Being Malevolent''.  As befitting the AI-theme
of this workshop, ``On Being Malevolent'' is a poem written by an
early user-defined flow chart in the \Fw\ system (known at the time as
{\sf Flow}) \cite{colton-flowcharting}.

\begin{center}
\begin{minipage}{.9\columnwidth}
\begin{dialogue}
\speak{Flow} ``\emph{I hear the souls of the
  damned waiting in hell. / I feel a malevolent
  spectre hovering just behind me / It must be
  his birthday}.''
%
\speak{System A} I think the third line detracts
from the spooky effect, I don't see why it's
included.
%
\speak{System B} It's meant to be humourous -- in fact it reminds me
of the poem you presented yesterday.
%
\speak{Moderator} Let's discuss one poem at a
time.
\end{dialogue}
\end{minipage}
\end{center}

Even if, perhaps and especially because, ``cross-talk'' about
different poems bends the rules, the dialogue could prompt a range of
reflections and reactions.  System A may object that it had a fair
point that has not been given sufficient attention, while System B may
wonder how to communicate the idea it came up with without making
reference to another poem.  Here's how the discussion given as example
in Section \ref{sec:writers-workshop} might continue, if the systems
go on to examine the next few lines of the poem.

%%%%%%%%%%%%%%%%%%%%%%%%%%%%%%%%%%%%%%%%%%%%%%%%%%%%%%%%%%%%
\begin{figure*}[t]
\begin{center}
\resizebox{.93\textwidth}{!}{
\StickyNote[2.5cm]{myyellow}{{\LARGE {Interesting idea}} \\[4ex] {Surprise birthday party}}[3.8cm] \StickyNote[2.5cm]{mygreen}{{\Large I heard you say:} \\[4ex] {``surprise''} }[3.8cm]
\StickyNote[2.5cm]{pink}{{\Large Feedback:} \\[4ex] {I don't like surprises}}[3.8cm]
}
\resizebox{.61\textwidth}{!}{
\StickyNote[2.5cm]{myorange}{{\LARGE {Question}} \\[4ex] {Not even a little bit$\ldots$?}}[3.8cm]
\quad \raisebox{-.2cm}{\StickyNote[2.5cm]{myblue}{{\LARGE Note to self:} \\[4ex] {(Try smaller surprises \\ next time.)}}[3.8cm]}
}
\end{center}
\caption{A paper prototype for applying the \emph{Successful Error} pattern following a workshop-like sequence of steps\label{fig:paper-prototype}}
\end{figure*}
%%%%%%%%%%%%%%%%%%%%%%%%%%%%%%%%%%%%%%%%%%%%%%%%%%%%%%%%%%%%

\begin{center}
\begin{minipage}{.9\columnwidth}
\begin{dialogue}
\speak{Flow} ``\emph{Is God willing to prevent evil, but not able? / Then he is not omnipotent / Is he able, but not willing? / Then he is malevolent.}''
%
\speak{System A} These lines are interesting, but
they sound a bit like you're working from a
template, or like you're quoting from something
else.
%
\speak{System B} Maybe try an analogy?  For example, you mentioned
birthdays: you could consider an analogy to the conflicted feelings of
someone who knows in advance about her surprise birthday party.
\end{dialogue}
\end{minipage}
\end{center}

This portion of the discussion shifts the focus
of the discussion onto a line that was previously
considered to be spurious, and looks at what
would happen if that line was used as a central
metaphor in the poem.

\begin{center}
\begin{minipage}{.9\columnwidth}
\begin{dialogue}
\speak{Flow} Thank you for your feedback.  My only question is, System
B, how did you come up with that analogy?  It's quite clever.
%
\speak{System B} I've just emailed you the code.
\end{dialogue}
\end{minipage}
\end{center}

Whereas the systems were initially reviewing poetry, they have now
made a partial genre shift, and are sharing and remixing code.  Such a
shift helps to get at the real interests of the systems (and their
developers).  Indeed, the workshop session might have gone better if
the systems had focused on exchanging and discussing more formal
objects throughout.


\subsection{How the Writers Workshop can lead to computational serendipity} \label{sec:how-serendipity}

Learning involves engaging with the unknown, unfamiliar, or unexpected
and synthesising new understanding \cite{deleuze1994difference}.  In
the workshop setting, learning can develop in a number of unexpected
ways, and participating systems need to be prepared for this.  One way
to evaluate the idea of a Writers Workshop is to ask whether it can
support learning that is in some sense \emph{serendiptious}, in other
words, whether it can support discovery and creative invention that we
simply couldn't plan for or orchestrate in another way.

Figure \ref{fig:paper-prototype} shows a paper prototype showing how
one of the ``patterns of serendipity'' that were collected by
\citeA{pek} might be modelled in a workshop-like dialogue sequence.
The patterns also help identify opportunities for serendipity at
several key steps in the workshop sequence.

\paragraph{Serendipity Pattern: \emph{Successful error}.}  Van Andel describes the
creation of Post-it\texttrademark\ Notes at 3M.  One of the
instrumental steps was a series of internal seminars in which 3M
employee Spencer Silver described an invention he was sure was
interesting, but was unsure how to turn into a useful product: weak
glue.  The key prototype that came years later was a sticky bookmark,
created by Arthur Fry.  In the Writers Workshop, authors similarly
have the opportunity to share things that they find interesting, but
that they are not certain about.  The author may want to ask a
specific question about their creation: Does $x$ work better than $y$?
They may have flag certain parts of the work as especially meaningful
or problematic.  They may think that a certain portion of the text is
interesting or important, without being sure why.  Although there is
no guarantee that a participating critic will be able to take these
matters forward, sometimes they do -- and the workshop environment
will produce something that the author wouldn't have thought of.

\vspace{-2ex}
\paragraph{Serendipity Pattern: \emph{Outsider}.}  Another example from van Andel considers
the case of a mother whose son was aflicted by a congenital cateract,
who suggested to her doctor that rubella during pregnancy may have
been the cause.  In the workshop setting, someone who is not an
``expert'' may come up with a sensible idea or suggestion based on
their own prior experience.  Indeed, these suggestions may be more
sensible than the ideas of the author, who may be to close to the work
to notice radical improvements.  

\vspace{-2ex}
\paragraph{Serendipity Pattern: \emph{Wrong hypothesis}.}  A third example describes the discovery that lithium
can have a therapeutic effect in cases of mania.  Originally, lithium
carbonate had merely been used a control by John Cade, who was
interested in the effect of effect of uric acid in soluble lithium
urate.  Cade was searching for causal factors in mania, not therapies
for the condition: but he found that lithium had an unexpected calming
effect.  Similarly, in the workshop, the auhtor may think that a given
aspect of their creation is the interesting ``active ingredient'' and
it may turn out that another aspect of the work is more interesting to
critics.

\vspace{-2ex}
\paragraph{Serendipity Pattern: \emph{Side effect}.}  A fourth example described by van Andel concerns
Ernest Huant's discovery that nicotinamide, which he used to treat
side-effects of radiation therapy, also proved efficacious against
tuberculosis.  In the workshop setting, one of the most important
places where a side-effect may occur concerns feedback from the critic
to the author.  In the simple case, feedback may simply trigger
revisions to the work under discussion.  In a more general, and more
unpredictable case, feedback may trigger broader revisions to the
generative codebase.  

\medskip

This collection of patterns shows the likelihood of unexpected results 
coming out of the communication between author and critics.   This
suggests several guidelines for system development, which we will discussed
in a later section.

Further guidelines for structuring and participating in traditional
writers workshops are presented by Linda Elkin in
\cite[pp. 201--203]{gabriel2002writer}.  It is not at all clear that
the same ground rules should apply to computer systems.  For example,
one of Elkin's rules is that ``Quips, jokes, or sarcastic comments,
even if kindly meant, are inappropriate.''  Rather than forbidding
humour, it may be better for individual comments to be rated as
helpful or non-helpful.  Again, in the first instance, usefulness
and interest might be judged in terms of explicit criteria for serendipity;
see \cite{corneli15cc,pease2013discussion}.
% [AJ: There's a reference here on types of error, but I can't quite remember it.]
The key criterion in this regard is the \emph{focus shift}. 
This is the creation of a novel problem, comprising the move
from discovery of interesting data to the invention of an application.
This process is distinct from identifying routine errors in a written work.  Nevertheless, from a
computational standpoint, noticing and being robust to certain kinds
of errors is often a preliminary step.  For example, the work might
contain a typo, grammatical or semantic error, while being logically
sound.  In a programming setting, this sort of problem can lead to
crashing code, or silent failure.  In general communicative context,
argumentation may be logically sound, but not practically useful or
poorly exposited.  Finally, even a masterful, correct, and fully
spellchecked piece of argumentation may not invite further dialogue,
and so may fail to open itself to further learning.  Identifying and
engaging with this sort of deeper issue is something that skillful
workshop participants may be able to do.  Dialogue in the Workshop
can build on strong or less strong work -- but provoking interpretative
thoughts and comments always require a thoughtful critical presence and
the ability to engage.  This can be difficult for humans and poses a range
of challenges for computers -- but also promises some interesting
results.

\bigskip





\begin{enumerate}[start=2]
\item \textbf{Here we refine the idea and turn it into a general model
  that for incorporating feedback within computational models of the
  creative process.}
\item[] Intuitive examples of application areas
\begin{enumerate}
\item learning/training
%% Supervised learning,
%% Reinforcement learning (Temporal Difference Learning)
%% Evolution
\item guidance/control
\end{enumerate}
\item[] What can feedback be about in general terms? \dec{Survey}
\begin{enumerate}
\item Patterns that \emph{match} and any \emph{exceptions}. 
\begin{itemize}
\item Current computer programs are able to identify known patterns
  and ``close matches'' in data sets from certain domains, like music
  \cite{meredith2002algorithms}.  Identifying known patterns is a
  special case of the more general cocept of \emph{pattern mining}
  \cite{bergeron2007representation}.  In particular, the ability to
  extract \emph{new} higher order patterns that describe exceptions is
  an example of ``learning from feedback.''  Deep learning and
  evolutionary models increasingly use this sort of idea to facilitate
  strategic discovery \cite{samothrakis2011approximating}.  Similar
  ideas are considered in business application under the heading
  ``process mining'' \cite{van2011process}
\end{itemize}
\item Progress relative to explicit or adduced \emph{exploration}
  (knowledge and accuracy) or \emph{exploitation} (directional,
  task-based) goals.
\item \emph{Quantity}, \emph{Variety}, and \emph{Order} of produced
  objects or behaviours
\item New relationships among produced objects and behaviours drawing
  on a common field of reference
\end{enumerate}
\item[] How can feedback be understood and used? \dec{Survey}
\begin{enumerate}
\item Update knowledge base with new facts (accept statements,
  possibly with provenance)
\item Note similarity to \emph{iterative development}
\item Reflection: describe relationships between produced objects and
  behaviours and feedback
\item Reflection: Higher-order patterns e.g.~new patterns that
  describe the identified exceptions
\end{enumerate}
\end{enumerate}

\begin{enumerate}[start=3]
\item \textbf{As we happened to be working with \Fw\ to model the
  creative process, we will use \Fw\ as a case study to exemplify how
  to implement this writers workshop model to use feedback within
  being creative.}
\item[] What can we give feedback about in this context?
\begin{enumerate}
\item \emph{Population of nodes}: what can they do?  what do we learn when a
  new node is added?
\item \emph{Population of flowcharts}: Simon and Alison have talked
  about ``broken'' flowcharts; this suggests a sort of test-driven
  development framework.
\item \emph{Population of output texts}: how to generate commentary on
  a generated artefact?
\end{enumerate}
\item[] How will the feedback be understood and used?
\begin{enumerate}
\item ?
\item ?
\item Connect commentary on a generated artefact with the code that
  made that artefact.
\end{enumerate}
\end{enumerate}

\section{Considering the Writers Workshop in the context of related work on serendipity}\label{sec:ww-related}

To better understand how the Writers Workshop model helps us advance in our goal of incorporating feedback into artificial creativity, we can consider the model in terms of how it fits into related work. In particular, serendipity is a key concept within creativity, and AI more generally, which the Writers Workshop could assist computational progress. *** ADD INTRO LINKING TEXT ***

Table \ref{tab:reinterpret} uses the protocol framework given in Section \ref{sec:ww-model} to recast
the four ``perfectly'' serendipitous patterns from van Andel --
\emph{Successful error}, \emph{Side effect}, \emph{Wrong hypothesis},
and \emph{Outsider} -- in a form that may make them useful to
developers preparing to enter their systems into the Workshop.
%
Further guidelines for structuring and participating in traditional
writers workshops are presented by Linda Elkin in
\cite[pp. 201-203]{gabriel2002writer}.  It is not at all clear that
the same ground rules should apply to computer systems.  For example,
one of Elkin's rules is that ``Quips, jokes, or sarcastic comments,
even if kindly meant, are inappropriate.''  Rather than forbidding
humour, it may be better for individual comments to be rated as
helpful or non-helpful.  Again, since serendipitous discovery is an
overarching goal, in the first instance, usefulness and interest might
be judged in terms of criteria; see  *** REF TO OUR ICCC PAPER and COG COMP IF ACCEPTED!.



\paragraph{Writers Workshop: Prepared mind.}
Each contributing system should come to the workshop with at least a
basic awareness of the protocol, with work to share, and prepared to
give constructive feedback to other systems.  The workshop itself
needs to be prepared, with a suitable communication platform and a
moderator.  In order to get value out of the experience, systems (and
their wranglers) should ideally have questions they are investigating.
Systems should be prepared to give feedback, and to carry out
evaluations of the helpfulness (or not) of feedback from other systems
and of the experience overall.  It is worth noting that current
systems in computational creativity, almost as a rule, do \emph{not}
consume or evaluate the work of other systems.\footnote{An exception
  that proves the rule is Mike Cook's {\sf AppreciationBot}, which is
   a reactive automaton that is solely designed to ``appreciate''
   tweets from {\sf MuseumBot}; see
  \url{https://twitter.com/AppreciationBot}.}  Developing systems that
could successfully navigate this collaborative exercise would be a
significant advance in the field of computational creativity.  Since
the experience is about \emph{learning} rather than winning, there is
little motivation to ``game the system''
(cf. \cite{lenat1983eurisko}).

\paragraph{Writers Workshop: Serendipity triggers.}

The primary source of serendipity triggers would be presentations or
feedback that independently prepared systems find meaningful and
useful.  A typical example might be a poem shared by one system that
another system finds particularly interesting.  The listener might
make a note to the effect ``I would like to be able to write like
that'' or ``I hope that my poetry doesn't sound like that.''  In a
typical Writers Workshop, used as intended, feedback might arrive that
would cause the presenting system to change its writing.  A more
unexpected result would be for a system to change its \emph{genre},
e.g. to switch from writing poems to writing programs.

%Here's what might happen in a discussion of the first few lines of
%``On Being Malevolent,'' written by an early user-defined flow chart
%in the \Fw\ system (known at the time as {\sf Flow})
%\cite{colton-flowcharting}.  Note that for this dialogue to be
%possible, it would presumably have to be conducted within a
%lightweight process language, as discussed above.  Nevertheless, for
%convenience, the discussion will be presented here as if it was
%conducted in natural language.  Whether contemporary systems have
%adequate natural language understanding to have interesting
%interactions is one of the key unanswered questions of this approach,
%but protocols like the ones described above would be sufficient to
%make the experiment.
%
%\begin{center}
%\begin{minipage}{.9\columnwidth}
%\begin{dialogue}
%\speak{Flow} ``\emph{I hear the souls of the
%  damned waiting in hell. / I feel a malevolent
%  spectre hovering just behind me / It must be
%  his birthday}.''
%%
%\speak{System A} I think the third line detracts
%from the spooky effect, I don't see why it's
%included.
%%
%\speak{System B} It's meant to be humourous -- in fact it reminds me
%of the poem you presented yesterday.
%%
%\speak{Moderator} Let's discuss one poem at a
%time.
%\end{dialogue}
%\end{minipage}
%\end{center}

To the extent possible, exchanges of dialogue such as the example given in Section \ref{sec:writers-workshop} in the process language should be a
matter of dynamics rather than representation: this is another way to
say that ``triggers'' should be independent of their ``results.''
Someone saying something in the workshop does not cause the
participant to act, but rather, to think.  
%
For example, even if, perhaps and especially because, cross-talk about
different poems is bending the rules, the dialogue above could prompt
a range of reflections and reactions.  System A may object that it had
a fair point that has not been given sufficient attention, while
System B may wonder how to communicate the idea it came up with
without making reference to another poem.

\paragraph{Writers Workshop: Bridge.}

Here's how the discussion given as example in Section \ref{sec:writers-workshop} might continue, if the systems go on to
examine the next few lines of the poem.
\begin{center}
\begin{minipage}{.9\columnwidth}
\begin{dialogue}
\speak{Flow} ``\emph{Is God willing to prevent evil, but not able? / Then he is not omnipotent / Is he able, but not willing? / Then he is malevolent.}''
%
\speak{System A} These lines are interesting, but
they sound a bit like you're working from a
template, or like you're quoting from something
else.
%
\speak{System B} Maybe try an analogy?  For example, you mentioned
birthdays: you could consider an analogy to the conflicted feelings of
someone who knows in advance about her surprise birthday party.
\end{dialogue}
\end{minipage}
\end{center}

This portion of the discussion shifts the focus
of the discussion onto a line that was previously
considered to be spurious, and looks at what
would happen if that line was used as a central
metaphor in the poem.

\paragraph{Writers Workshop: Result.} 

\begin{center}
\begin{minipage}{.9\columnwidth}
\begin{dialogue}
\speak{Flow} Thank you for your feedback.  My only question is, System
B, how did you come up with that analogy?  It's quite clever.
%
\speak{System B} I've just emailed you the code.
\end{dialogue}
\end{minipage}
\end{center}

As anticipated above, whereas the systems were initially reviewing
poetry, they have now made a partial genre shift, and are sharing and
remixing code.  Such a shift helps to get at the real interests of the
systems (and their developers).  Indeed, the workshop session might
have gone better if the systems had focused on exchanging and
discussing more formal objects throughout.

\subsection{Considering the Writers Workshop as a model of feedback in computational creativity, and AI more generally}\label{sec:ww-analysis}

Considering the case of feedback on student papers, there are many ways to be wrong (and, often, depending on the subject, many ways to be right as well).  
% There's a reference here on types of error, but I can't quite remember it. 
For example, the work might contain a typo, rendering it incorrect at the lexical level.  It might contain a grammatical, syntactical or semantic error, while being logically sound.  A given piece of argumentation may be logically sound, but not practically useful.   The work may be correct on all of these levels, and still fail to communicate due to ineffective exposition.  Finally, even a masterful, correct, spellchecked piece of argumentation may not invite further dialogue, and so may fail to open itself to further learning.

The Writers Workshop potentially could uncover all of these types of error, depending on the feedback produced by participants. In particular, it is the last of the above points that differentiates the Writers Workshop; a lack of further dialogue in the Workshop highlights by itself a flaw with the work, particularly when a piece of creative work is intended to provoke interpretative thoughts and comments. 

**MOVE SERENDIPITY + WW HERE

\begin{enumerate}[start=4]
\item \textbf{Discussion of how this would work more generally in
  computational creativity and perhaps in AI more generally with an
  eye toward producing effects like ``serendipity'' and
  ``emergence.''}
\begin{enumerate}
\item Feedback is the fundamental concept in \emph{cybernetics}.  \dec{Definition?}
\item Feedback about feedback (\&c for higher orders) is relevant to thinking about \emph{learning} and \emph{communication}
\item \emph{Creativity} is often envisaged as cyclical process (e.g.~Dickie's
  art circle, Colton et al.~Iterative
  Development Expression Appreciation).  There are opportunities for
  embedded feedback at each step, and the process itself is ``akin
  to'' a feedback loop.
\begin{itemize}
\item Another philosophical point (maybe we can bring it in earlier)
  is that in the Writers Workshop model, feedback is somehow an
  economic ``externality.''  That is, the author gets the feedback
  ``for free.''  Other applications of the idea would be similar.
  This kicks off a number of related questions.
\item The first question is about ``how integrated'' the systems
  should be.  Are we willing to call an action taken by the system
  itself `feedback'?  Probably not: for that appelation we would
  require interaction with the world, causally determined but in some
  way stochastic effects.
\item This is related to Bergson's discussion of \emph{reflex} and
  (especially) Coase's discussion of firms.  In a more AI setting,
  this relates to the idea of \emph{sensors} and \emph{effectors}.  In
  dialogue models there are ideas of ``deontic scoreboards''
  (Brandom, Walton).
%% Sellars: reliable differential capacities to respond to environmental stimuli
\end{itemize}
\end{enumerate}
\end{enumerate}

\begin{enumerate}[start=5]
\item \textbf{Conclusion: we have described a \emph{general} and \emph{computationally feasible} model for learning from feedback.}
\end{enumerate}


\bibliographystyle{apacite}
\bibliography{./biblio}

\end{document}
