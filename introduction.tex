\section{Introduction} \label{sec:introduction}

%AI and Feedback is the first international workshop on the topic and it is one of the IJCAI 2015 workshops. It focuses on applying AI techniques for addressing the challenges of mining and extracting feedback, as well as assessing, analysing, and making use of feedback.
%
%Feedback is key for both improvement and decision making. As humans, we are designed to constantly seek feedback on how and what we are doing in life. Feedback can come from ourselves, from our peers, from our teachers, from our collaborators, audiences, customers, public or press. Feedback provides opportunities to learn about how we and our work are perceived by others. If we encounter someone [something] new, we can examine previous feedback to learn how this new person [thing] is perceived by others.
%
%A key target of this workshop is to discuss how to build intelligent feedback agents that are capable of autonomously providing feedback that equals or surpasses that of human beings in its usefulness. The feedback of artificial feedback agents should have some desirable characteristics. It should be socially and culturally appropriate, clearly expressed, sufficiently focused and contextualised, thoughtfully challenging yet encouraging, compassionate, open to debate, justified and comparative, also, it should be trustworthy. Giving and receiving feedback with these characteristics therefore is a challenging, creative process.
%
%Aims and objectives. This workshop aims to bring together researchers from different strands of AI to discuss various aspects of feedback. The following questions are examples for consideration:

% * How can we build creative feedback agents that can generate feedback, especially on creative work?
% * How can we provide a sufficient agency in AI systems so that feedback from such systems will be useful?
% * How can we model feedback so that it can be generated by AI systems?
% * How can we design intuitive environments in which groups of humans and AI systems can give each other feedback?
% * How can we design systems that can act as creative collaborators? That not only give feedback, but can actually propose changes to a developing artefact?
% * How can we mine and learn from large datasets of feedback, e.g. educational social networks?
% * How can implicit feedback be extracted and interpreted?
% * How can trust and reputation models help with the assessment of feedback?
% * How can we process and use feedback?
% * What are the possible applications for AI systems capable of giving creative feedback?

%Topics of Interest
%
%Topics of interest cover a variety of feedback-related issues, such as mining and extracting feedback, generating feedback, understanding feedback, and assessing feedback.
%
%Topics of interest cover different strands of AI, such as multiagent systems, machine learning, natural language processing, and knowledge representation.
%
%Topics include, but are not limited to:
%
%Ontologies of feedback
%Multimodal feedback
%Implicit feedback
%Opinion mining and sentiment analysis
%Automatic generation of feedback
%Modelling the impact of feedback
%Designing environments for feedback
%Feedback and machine learning
%Trust and reputation models for feedback analysis
%Applications: creative industries, music composition, online learning, etc.

%We welcome and strongly encourage the submission of high quality, original work (published or unpublished), as well as visionary papers and roadmaps relevant to the scope of this workshop.
%
%Submitted papers must be formatted according to IJCAI guidelines. Formatting guidelines and electronic templates are available here.
%
%Submitted papers should not exceed 8 pages, excluding the bibliographic references.

In educational applications it would be useful to have an automated tutor that can read student work and make suggestions based on diagnostics, like, is the paper wrong, and if so how?  What background material should be recommended to the student for review?

In the current paper, we ``flip the script'' and look at what we believe to be a more fundamental problem for AI: computer programs that can themselves learn from feedback.  After all, if it was easy to build great automatic tutors, they would be a part of everyday life.  We look forward to a future when that is the case.

Computational creativity is a challenge within artificial intelligence where feedback plays a vital part [REFS***] \cite{perezyperez10MM,pease10}. Creativity cannot happen in a `silo' but instead is influenced and affected by feedback and interaction with others \cite{csik88,saunders2012towards}. Computational creativity researchers are starting to place more emphasis on social interaction and feedback by computational systems [***EVIDENCE ***]. Still, nearly 3 in 4 papers at the 2014 International Conference for Computational Creativity\footnote{ICCC is  the key international conference for research in computational creativity.} failed to acknowledge the role of feedback or social communication in their computational work on creativity. 

To highlight and contribute towards modelling feedback as a crucial part of creativity, we propose in this paper a model of computational feedback for creative systems based on Writers Workshops \cite{gabriel2002writer}, a literary collaborative practice that encourages interactive feedback within the creative process. We introduce Writers Workshops (Section \ref{sec:writers-workshop}), discuss their usefulness as a candidate for a computational model of feedback (Section \ref{sec:ww-analysis}) and propose such a model (Section \ref{sec:ww-model}). We partially evaluate our model by reflecting on how it fits with related work encouraging serendipity and emergence in computational models of intelligence and creativity \ref{sec:ww-related}. 
While we acknowledge that this paper is offering a roadmap for this model rather than a full implementation, we consider how the model could be practically implemented in a computational system and report our initial implementation work (Section \ref{sec:implementation}). We conclude by looking at ways in which this work can be usefully directed in the future.

%Current computer programs are able to identify patterns and ``close matches'' in data sets from certain domains, like music (David Meredith).  Learning \emph{new} patterns on the fly is harder but potentially quite useful. 
% AJ Don't see what this adds?


%Before we can learn from AI systems, we will have to teach them -- and learn how to learn together with them.  Accordingly, we \emph{design for emergence}.  (Say more.)
% AJ in this community, we need to be careful talking about emergence. It's been studied a lot and we'd need to reference a whole heap of work... Hence a suggested change in title below and above
