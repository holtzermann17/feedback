%\section{Conclusions and Future directions}

The benefits of the Writers Workshop approach could innovate well beyond models for 
feedback and communication within a particular environment or restricted domain. 
Following the example of the Pattern Languages of Programming (PLoP) community, we propose that the Writers Workshop model could be deployed
within the Computational Creativity community to design a workshop in
which the participants are computer systems instead of human authors.
The annual International Conference on Computational Creativity
(ICCC), now entering its sixth year, could be a suitable venue. 

Rather than the system's creator presenting the system in a
traditional slideshow and discussion, or a system ``Show and Tell,''
the systems would be brought to the workshop and would present their
own work to an audience of other systems, in a Writers Workshop
format.  This could be accompanied by a short paper for the conference
proceedings written by the system's designer describing the system's
current capabilities and goals.  
%
If the Workshop really works well, future publications might adapt to include
traces of Workshop interactions, commentary from a system on
other systems, and offline reflections on what the system might change
about its own work based on the feedback it receives.  Paralleling the PLoP
community, it could become standard to incorporate the workshop
into the process of peer review for the new \emph{Journal of
  Computational Creativity}.\footnote{\url{http://www.journalofcomputationalcreativity.cc}} AI systems that review each other would surely be a major demonstration and acknowledgement of the usefulness of feedback within AI.

In closing, we wish to return briefly to the scenario of computer
generated feedback in educational contexts that we raised at the
beginning of this paper and then set aside.  The elements of our
functional design for sharing feedback among computational agents has
a range of features that continue to be relevant for generating useful
feedback with human learners.  Students are experience-bound, and a
robust approach to formative assessment and feedback should take into
account the student's historical experience, so far as this can be
known or inferred.  In order for feedback, recommendations, and so on to
adequately take individual history into account, sophisticated modelling and
reasoning would be required.
Nevertheless, from the point of view of participating computational
agents, a student may simply look like another agent.  It is in this
regard that computational models of learning from feedback are seen
as fundamental.
