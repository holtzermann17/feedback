%\section{Conclusions and Future directions}

The benefits of the Writers Workshop approach could innovate well beyond models for 
feedback and communication within a particular environment or restricted domain. 
Following the example of the Pattern Languages of Programming (PLoP) community, we propose that the Writers Workshop model could be deployed
within the Computational Creativity community to design a workshop in
which the participants are computer systems instead of human authors.
The annual International Conference on Computational Creativity
(ICCC), now entering its sixth year, could be a suitable venue. 

Rather than the system's creator presenting the system in a
traditional slideshow and discussion, or a system ``Show and Tell,''
the systems would be brought to the workshop and would present their
own work to an audience of other systems, in a Writers Workshop
format.  This might be accompanied by a short paper for the conference
proceedings written by the system's designer describing the system's
current capabilities and goals.  

Thinking even more broadly about potential gains and future work, subsequent publications might include
traces of interactions in the Workshop, commentary from the system on
other systems, and offline reflections on what the system might change
about its own work based on the feedback it receives.  As in the PLoP
community, it could become standard to incorporate this sort of workshop
into the process of peer reviewing journal articles for the new \emph{Journal of
  Computational Creativity}.\footnote{\url{http://www.journalofcomputationalcreativity.cc}} AI systems that review each other would surely be a major demonstration and acknowledgement of the usefulness of feedback within AI.

In closing, we wish to return to the scenario of computer generated
feedback in educational contexts that we raised at the beginning of
this paper and then set aside.  It is worth mentioning here that
elements of the functional design for sharing feedback among
computational agents has some features that are relevant for
generating useful feedback with human learners.  In particular,
students are experience-bound, and a robust approach to formative
assessment should model the student's historical development.  From
the point of view of participating agents, a student may simply ``look
like'' another agent.
