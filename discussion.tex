\section{Considering the Writers Workshop in the context of related work on serendipity}\label{sec:ww-related}

To better understand how the Writers Workshop model helps us advance in our goal of incorporating feedback into artificial creativity, we can consider the model in terms of how it fits into related work. In particular, serendipity is a key concept within creativity, and AI more generally, which the Writers Workshop could assist computational progress. *** ADD INTRO LINKING TEXT ***

Table \ref{tab:reinterpret} uses the protocol framework given in Section \ref{sec:ww-model} to recast
the four ``perfectly'' serendipitous patterns from van Andel --
\emph{Successful error}, \emph{Side effect}, \emph{Wrong hypothesis},
and \emph{Outsider} -- in a form that may make them useful to
developers preparing to enter their systems into the Workshop.
%
Further guidelines for structuring and participating in traditional
writers workshops are presented by Linda Elkin in
\cite[pp. 201-203]{gabriel2002writer}.  It is not at all clear that
the same ground rules should apply to computer systems.  For example,
one of Elkin's rules is that ``Quips, jokes, or sarcastic comments,
even if kindly meant, are inappropriate.''  Rather than forbidding
humour, it may be better for individual comments to be rated as
helpful or non-helpful.  Again, since serendipitous discovery is an
overarching goal, in the first instance, usefulness and interest might
be judged in terms of criteria; see  *** REF TO OUR ICCC PAPER and COG COMP IF ACCEPTED!.

%\paragraph{Writers Workshop: Prepared mind.}
Each contributing system should come to the workshop with at least a
basic awareness of the protocol, with work to share, and prepared to
give constructive feedback to other systems.  The workshop itself
needs to be prepared, with a suitable communication platform and a
moderator.  In order to get value out of the experience, systems (and
their wranglers) should ideally have questions they are investigating.
Systems should be prepared to give feedback, and to carry out
evaluations of the helpfulness (or not) of feedback from other systems
and of the experience overall.  It is worth noting that current
systems in computational creativity, almost as a rule, do \emph{not}
consume or evaluate the work of other systems.\footnote{An exception
  that proves the rule is Mike Cook's {\sf AppreciationBot}, which is
   a reactive automaton that is solely designed to ``appreciate''
   tweets from {\sf MuseumBot}; see
  \url{https://twitter.com/AppreciationBot}.}  Developing systems that
could successfully navigate this collaborative exercise would be a
significant advance in the field of computational creativity.  Since
the experience is about \emph{learning} rather than winning, there is
little motivation to ``game the system''
(cf. \cite{lenat1983eurisko}).

%\paragraph{Writers Workshop: Serendipity triggers.}
The primary source of serendipity triggers would be presentations or
feedback that independently prepared systems find meaningful and
useful.  A typical example might be a poem shared by one system that
another system finds particularly interesting.  The listener might
make a note to the effect ``I would like to be able to write like
that'' or ``I hope that my poetry doesn't sound like that.''  In a
typical Writers Workshop, used as intended, feedback might arrive that
would cause the presenting system to change its writing.  A more
unexpected result would be for a system to change its \emph{genre},
e.g. to switch from writing poems to writing programs.

\begin{table}[p]
\begin{tabular}{p{.95\columnwidth}}
{\bf\emph{Successful error}}  \\
\emph{Van Andel's example}:  Post-it\texttrademark\ notes \\[.2cm]
{\tt presentation} Systems should be prepared to share interesting ideas even if they don't know directly how they will be useful.  \\
{\tt listening}    Systems should listen with interest, too. \\
{\tt feedback}     Even interesting ideas may not be ``marketable.''\\
{\tt questions}    How is your suggestion useful? \\
{\tt reflections}  New combinations of ideas take a long time to realise, and many different ideas may need to be combined in order to come up with something useful.\\
\end{tabular}
\medskip

\begin{tabular}{p{.95\columnwidth}}
{\bf\emph{Side effect}}  \\
\emph{Van Andel's example}:  Nicotinamide used to treat side-effects of radiation therapy proves efficacious against tuberculosis. \\[.2cm]
{\tt presentation} Systems should use their presentation as an experiment. \\
{\tt listening}    Listeners should allow themselves to be affected by what they are hearing. \\
{\tt feedback}     Feedback should convey the nature of the effect.\\
{\tt questions}    The presenter may need to ask follow-up questions to gain insight. \\
{\tt reflections}  Form a new hypothesis before seeking a new audience. \\
\end{tabular}
\medskip

\begin{tabular}{p{.95\columnwidth}}
{\bf\emph{Wrong hypothesis}}  \\
\emph{Van Andel's example}:  Lithium, used in a control study, had an unexpected calming effect. \\[.2cm]
{\tt presentation} How is this presentation interpretable as a (``natural'') control study? \\
{\tt listening}    Listeners are ``guinea pigs''.\\
{\tt feedback}     Discuss side-effects that do not necessarily correspond to the author's perceived intent. \\
{\tt questions}    Zero in on the most interesting part of the conversation.\\
{\tt reflections}  Revise hypotheses to correspond to the most surprising feedback. \\
\end{tabular}
\medskip

\begin{tabular}{p{.95\columnwidth}}
{\bf\emph{Outsider}}  \\
\emph{Van Andel's example}:  A mother suggests a new hypothesis to a doctor. \\[.2cm]
{\tt presentation} The presenter is here to learn from the audience. \\
{\tt listening}   The audience is here to give help, but also to get help.\\
{\tt feedback}     Feedback will inevitably draw on previous experiences and ideas.\\
{\tt questions}    What is the basis for that remark?\\
{\tt reflections}  How can I implement the suggestions?\\
\end{tabular}
\vspace{.2cm}
\caption{Reinterpreting patterns of serendipity for use in a computational workshop\label{tab:reinterpret}}
\end{table}

% \subsection{Considering the Writers Workshop as a model of feedback in computational creativity, and AI more generally}\label{sec:ww-analysis}

\bigskip

Considering the case of feedback on student papers, there are many ways to be wrong (and, often, depending on the subject, many ways to be right as well).  
% There's a reference here on types of error, but I can't quite remember it. 
For example, the work might contain a typo, rendering it incorrect at the lexical level.  It might contain a grammatical, syntactical or semantic error, while being logically sound.  A given piece of argumentation may be logically sound, but not practically useful.   The work may be correct on all of these levels, and still fail to communicate due to ineffective exposition.  Finally, even a masterful, correct, spellchecked piece of argumentation may not invite further dialogue, and so may fail to open itself to further learning.

The Writers Workshop potentially could uncover all of these types of error, depending on the feedback produced by participants. In particular, it is the last of the above points that differentiates the Writers Workshop; a lack of further dialogue in the Workshop highlights by itself a flaw with the work, particularly when a piece of creative work is intended to provoke interpretative thoughts and comments. 

**MOVE SERENDIPITY + WW HERE

\begin{enumerate}[start=4]
\item \textbf{Discussion of how this would work more generally in
  computational creativity and perhaps in AI more generally with an
  eye toward producing effects like ``serendipity'' and
  ``emergence.''}
\begin{enumerate}
\item Feedback is the fundamental concept in \emph{cybernetics}.  \dec{Definition?}
\item Feedback about feedback (\&c for higher orders) is relevant to thinking about \emph{learning} and \emph{communication}
\item \emph{Creativity} is often envisaged as cyclical process (e.g.~Dickie's
  art circle, Colton et al.~Iterative
  Development Expression Appreciation).  There are opportunities for
  embedded feedback at each step, and the process itself is ``akin
  to'' a feedback loop.
\begin{itemize}
\item Another philosophical point (maybe we can bring it in earlier)
  is that in the Writers Workshop model, feedback is somehow an
  economic ``externality.''  That is, the author gets the feedback
  ``for free.''  Other applications of the idea would be similar.
  This kicks off a number of related questions.
\item The first question is about ``how integrated'' the systems
  should be.  Are we willing to call an action taken by the system
  itself `feedback'?  Probably not: for that appelation we would
  require interaction with the world, causally determined but in some
  way stochastic effects.
\item This is related to Bergson's discussion of \emph{reflex} and
  (especially) Coase's discussion of firms.  In a more AI setting,
  this relates to the idea of \emph{sensors} and \emph{effectors}.  In
  dialogue models there are ideas of ``deontic scoreboards''
  (Brandom, Walton).
%% Sellars: reliable differential capacities to respond to environmental stimuli
\end{itemize}
\end{enumerate}
\end{enumerate}
