\section{Concluding discussion and future directions}

 We have described a \emph{general} and \emph{computationally
   feasible} model for using feedback in AI systems, particularly
 creative systems.  The Writers Workshop concept, borrowed from
 creative writing, is transformed into a model of a structured
 approach to eliciting, processing and learning from feedback.  To
 better evaluate how the Writers Workshop model helps us advance in
 our goal of incorporating feedback into artificial creativity, we
 critically considered how the model fits into related work. In
 particular, we found that serendipity, a key concept within
 creativity and AI more generally, is a concept with which the Writers
 Workshop model could assist computational progress.
In this respect, we should highlight the difference between
``global'' analytics describing the collection of nodes and
flowcharts in the FloWr ecosystem, and the path-dependent
analysis that takes place in a workshop setting.

Our preliminary implementation work (Section \ref{sec:implementation}) shows
that the model can be transfered to a functional implementation.  This
work highlights several considerations relevant to further work with
the Writers Workshop model:

\begin{itemize}
\item Each contributing system should come to the workshop with at
  least a basic awareness of the workshop protocol, with work to
  share, and prepared to give constructive feedback to other systems.
\item The workshop itself needs to be prepared, with a suitable
  communication platform and a moderator or global flowchart for
  moving the discussion from step to step.
 \item A controlled vocabulary for communications and interaction
   would be a worthwhile pursuit of future research, perhaps based on
   an ontology, inspired by the Interaction Network
   Ontology.\footnote{The Interaction Network Ontology describes
     interactions within humans as opposed to within human societies;
     a \emph{Social} Interaction Network Ontology does not seem to
     exist at present.  The Interaction Network Ontology is documented
     at \url{http://www.ontobee.org/browser/index.php?o=INO}.  Its URI
     is
     \url{http://svn.code.sf.net/p/ino/code/trunk/src/ontology/INO.owl}.}
\item In order to get the most value out of the workshop experience,
  systems (and their wranglers) should ideally have questions they are
  investigating.  As discussed above, prior experience plays an
  important role in every step.  This opens up a range of issues for
  further research on modeling motivations and learning from
  experience.
\item Systems should be prepared to give feedback, and to carry out
  evaluations of the helpfulness (or not) of feedback from other
  systems and of the experience overall.
\end{itemize}
  
  Developing systems that could successfully navigate this
  collaborative exercise would be a significant advance in the field
  of computational creativity.  Since the experience is about learning
  rather than winning, there is little motivation to ``game the
  system'' (cf. \citeNP{lenat1983eurisko}). Instead the emphasis is
  squarely upon mutual benefit: computational systems helping to
  develop each other through communication and feedback.
