\section{Considering the Writers Workshop in the context of related work on serendipity}\label{sec:ww-related}

To better understand how the Writers Workshop model helps us advance in our goal of incorporating feedback into artificial creativity, we can consider the model in terms of how it fits into related work. In particular, serendipity is a key concept within creativity, and AI more generally, which the Writers Workshop could assist computational progress. *** ADD INTRO LINKING TEXT ***

Table \ref{tab:reinterpret} uses the protocol framework given in Section \ref{sec:ww-model} to recast
the four ``perfectly'' serendipitous patterns from van Andel --
\emph{Successful error}, \emph{Side effect}, \emph{Wrong hypothesis},
and \emph{Outsider} -- in a form that may make them useful to
developers preparing to enter their systems into the Workshop.
%
Further guidelines for structuring and participating in traditional
writers workshops are presented by Linda Elkin in
\cite[pp. 201-203]{gabriel2002writer}.  It is not at all clear that
the same ground rules should apply to computer systems.  For example,
one of Elkin's rules is that ``Quips, jokes, or sarcastic comments,
even if kindly meant, are inappropriate.''  Rather than forbidding
humour, it may be better for individual comments to be rated as
helpful or non-helpful.  Again, since serendipitous discovery is an
overarching goal, in the first instance, usefulness and interest might
be judged in terms of criteria; see  *** REF TO OUR ICCC PAPER and COG COMP IF ACCEPTED!.



\paragraph{Writers Workshop: Prepared mind.}
Each contributing system should come to the workshop with at least a
basic awareness of the protocol, with work to share, and prepared to
give constructive feedback to other systems.  The workshop itself
needs to be prepared, with a suitable communication platform and a
moderator.  In order to get value out of the experience, systems (and
their wranglers) should ideally have questions they are investigating.
Systems should be prepared to give feedback, and to carry out
evaluations of the helpfulness (or not) of feedback from other systems
and of the experience overall.  It is worth noting that current
systems in computational creativity, almost as a rule, do \emph{not}
consume or evaluate the work of other systems.\footnote{An exception
  that proves the rule is Mike Cook's {\sf AppreciationBot}, which is
   a reactive automaton that is solely designed to ``appreciate''
   tweets from {\sf MuseumBot}; see
  \url{https://twitter.com/AppreciationBot}.}  Developing systems that
could successfully navigate this collaborative exercise would be a
significant advance in the field of computational creativity.  Since
the experience is about \emph{learning} rather than winning, there is
little motivation to ``game the system''
(cf. \cite{lenat1983eurisko}).

\paragraph{Writers Workshop: Serendipity triggers.}

The primary source of serendipity triggers would be presentations or
feedback that independently prepared systems find meaningful and
useful.  A typical example might be a poem shared by one system that
another system finds particularly interesting.  The listener might
make a note to the effect ``I would like to be able to write like
that'' or ``I hope that my poetry doesn't sound like that.''  In a
typical Writers Workshop, used as intended, feedback might arrive that
would cause the presenting system to change its writing.  A more
unexpected result would be for a system to change its \emph{genre},
e.g. to switch from writing poems to writing programs.

%Here's what might happen in a discussion of the first few lines of
%``On Being Malevolent,'' written by an early user-defined flow chart
%in the \Fw\ system (known at the time as {\sf Flow})
%\cite{colton-flowcharting}.  Note that for this dialogue to be
%possible, it would presumably have to be conducted within a
%lightweight process language, as discussed above.  Nevertheless, for
%convenience, the discussion will be presented here as if it was
%conducted in natural language.  Whether contemporary systems have
%adequate natural language understanding to have interesting
%interactions is one of the key unanswered questions of this approach,
%but protocols like the ones described above would be sufficient to
%make the experiment.
%
%\begin{center}
%\begin{minipage}{.9\columnwidth}
%\begin{dialogue}
%\speak{Flow} ``\emph{I hear the souls of the
%  damned waiting in hell. / I feel a malevolent
%  spectre hovering just behind me / It must be
%  his birthday}.''
%%
%\speak{System A} I think the third line detracts
%from the spooky effect, I don't see why it's
%included.
%%
%\speak{System B} It's meant to be humourous -- in fact it reminds me
%of the poem you presented yesterday.
%%
%\speak{Moderator} Let's discuss one poem at a
%time.
%\end{dialogue}
%\end{minipage}
%\end{center}

To the extent possible, exchanges of dialogue such as the example given in Section \ref{sec:writers-workshop} in the process language should be a
matter of dynamics rather than representation: this is another way to
say that ``triggers'' should be independent of their ``results.''
Someone saying something in the workshop does not cause the
participant to act, but rather, to think.  
%
For example, even if, perhaps and especially because, cross-talk about
different poems is bending the rules, the dialogue above could prompt
a range of reflections and reactions.  System A may object that it had
a fair point that has not been given sufficient attention, while
System B may wonder how to communicate the idea it came up with
without making reference to another poem.

\paragraph{Writers Workshop: Bridge.}

Here's how the discussion given as example in Section \ref{sec:writers-workshop} might continue, if the systems go on to
examine the next few lines of the poem.
\begin{center}
\begin{minipage}{.9\columnwidth}
\begin{dialogue}
\speak{Flow} ``\emph{Is God willing to prevent evil, but not able? / Then he is not omnipotent / Is he able, but not willing? / Then he is malevolent.}''
%
\speak{System A} These lines are interesting, but
they sound a bit like you're working from a
template, or like you're quoting from something
else.
%
\speak{System B} Maybe try an analogy?  For example, you mentioned
birthdays: you could consider an analogy to the conflicted feelings of
someone who knows in advance about her surprise birthday party.
\end{dialogue}
\end{minipage}
\end{center}

This portion of the discussion shifts the focus
of the discussion onto a line that was previously
considered to be spurious, and looks at what
would happen if that line was used as a central
metaphor in the poem.

\paragraph{Writers Workshop: Result.} 

\begin{center}
\begin{minipage}{.9\columnwidth}
\begin{dialogue}
\speak{Flow} Thank you for your feedback.  My only question is, System
B, how did you come up with that analogy?  It's quite clever.
%
\speak{System B} I've just emailed you the code.
\end{dialogue}
\end{minipage}
\end{center}

As anticipated above, whereas the systems were initially reviewing
poetry, they have now made a partial genre shift, and are sharing and
remixing code.  Such a shift helps to get at the real interests of the
systems (and their developers).  Indeed, the workshop session might
have gone better if the systems had focused on exchanging and
discussing more formal objects throughout.

\subsection{Considering the Writers Workshop as a model of feedback in computational creativity, and AI more generally}\label{sec:ww-analysis}

Considering the case of feedback on student papers, there are many ways to be wrong (and, often, depending on the subject, many ways to be right as well).  
% There's a reference here on types of error, but I can't quite remember it. 
For example, the work might contain a typo, rendering it incorrect at the lexical level.  It might contain a grammatical, syntactical or semantic error, while being logically sound.  A given piece of argumentation may be logically sound, but not practically useful.   The work may be correct on all of these levels, and still fail to communicate due to ineffective exposition.  Finally, even a masterful, correct, spellchecked piece of argumentation may not invite further dialogue, and so may fail to open itself to further learning.

The Writers Workshop potentially could uncover all of these types of error, depending on the feedback produced by participants. In particular, it is the last of the above points that differentiates the Writers Workshop; a lack of further dialogue in the Workshop highlights by itself a flaw with the work, particularly when a piece of creative work is intended to provoke interpretative thoughts and comments. 

**MOVE SERENDIPITY + WW HERE

\begin{enumerate}[start=4]
\item \textbf{Discussion of how this would work more generally in
  computational creativity and perhaps in AI more generally with an
  eye toward producing effects like ``serendipity'' and
  ``emergence.''}
\begin{enumerate}
\item Feedback is the fundamental concept in \emph{cybernetics}.  \dec{Definition?}
\item Feedback about feedback (\&c for higher orders) is relevant to thinking about \emph{learning} and \emph{communication}
\item \emph{Creativity} is often envisaged as cyclical process (e.g.~Dickie's
  art circle, Colton et al.~Iterative
  Development Expression Appreciation).  There are opportunities for
  embedded feedback at each step, and the process itself is ``akin
  to'' a feedback loop.
\begin{itemize}
\item Another philosophical point (maybe we can bring it in earlier)
  is that in the Writers Workshop model, feedback is somehow an
  economic ``externality.''  That is, the author gets the feedback
  ``for free.''  Other applications of the idea would be similar.
  This kicks off a number of related questions.
\item The first question is about ``how integrated'' the systems
  should be.  Are we willing to call an action taken by the system
  itself `feedback'?  Probably not: for that appelation we would
  require interaction with the world, causally determined but in some
  way stochastic effects.
\item This is related to Bergson's discussion of \emph{reflex} and
  (especially) Coase's discussion of firms.  In a more AI setting,
  this relates to the idea of \emph{sensors} and \emph{effectors}.  In
  dialogue models there are ideas of ``deontic scoreboards''
  (Brandom, Walton).
%% Sellars: reliable differential capacities to respond to environmental stimuli
\end{itemize}
\end{enumerate}
\end{enumerate}
