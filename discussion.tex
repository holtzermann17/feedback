\section{Critiquing the Writers Workshop in computational contexts}\label{sec:ww-related}

Considering the case of feedback on student papers, there are many ways to be wrong (and, often, depending on the subject, many ways to be right as well).  
% There's a reference here on types of error, but I can't quite remember it. 
For example, the work might contain a typo, rendering it incorrect at the lexical level.  It might contain a grammatical, syntactical or semantic error, while being logically sound.  A given piece of argumentation may be logically sound, but not practically useful.   The work may be correct on all of these levels, and still fail to communicate due to ineffective exposition.  Finally, even a masterful, correct, and fully spellchecked piece of argumentation may not invite further dialogue, and so may fail to open itself to further learning.

The Writers Workshop potentially could uncover all of these types of error, depending on the feedback produced by participants. In particular, it is the last of the above points that differentiates the Writers Workshop; a lack of further dialogue in the Workshop highlights by itself a flaw with the work, particularly when a piece of creative work is intended to provoke interpretative thoughts and comments. 

\subsection{Writers Workshop and computational serendipity}

To better evaluate how the Writers Workshop model helps us advance in our goal of incorporating feedback into artificial creativity, we can consider the model in terms of how it fits into related work. In particular, serendipity is a key concept within creativity, and AI more generally, which the Writers Workshop could assist computational progress. 

Each contributing system should come to the workshop with at least a
basic awareness of the protocol, with work to share, and prepared to
give constructive feedback to other systems.  The workshop itself
needs to be prepared, with a suitable communication platform and a
moderator.  In order to get value out of the experience, systems (and
their wranglers) should ideally have questions they are investigating.
Systems should be prepared to give feedback, and to carry out
evaluations of the helpfulness (or not) of feedback from other systems
and of the experience overall.  It is worth noting that current
systems in computational creativity, almost as a rule, do \emph{not}
consume or evaluate the work of other systems.\footnote{An exception
  that proves the rule is Mike Cook's {\sf AppreciationBot}, which is
  a reactive automaton that is solely designed to ``appreciate''
  tweets from {\sf MuseumBot}; see
  \url{https://twitter.com/AppreciationBot}.}  Developing systems that
could successfully navigate this collaborative exercise would be a
significant advance in the field of computational creativity.  Since
the experience is about learning rather than winning, there is little
motivation to ``game the system'' (cf. \citeNP{lenat1983eurisko}).

Learning involves engaging with the unknown, unfamiliar, or unexpected
and synthesising new understanding \cite{deleuze1994difference}.  In
this regard learning is somewhat \emph{serendiptious}, even when it is
planned as part of a course of study or another learning exercise.  In the
workshop setting, learning can develop in a number of unexpected
ways, and participating systems need to be prepared for this.

Table \ref{tab:reinterpret} adapts four patterns of serendipity from
\citeNP{pek} -- \emph{Successful error}, \emph{Side effect},
\emph{Wrong hypothesis}, and \emph{Outsider} -- in a form that may
make them useful to developers preparing to enter their systems into
the Workshop.  This is a small seed collection of heuristics that
would complement basic knowledge relevant to the artefacts that are
discussed, e.g., for poetry, textual features like repetition and
narrative \cite{corneli15iccc}.  In a typical Writers Workshop,
feedback will inspire the presenter to revise a draft of work in
progress.  A potentially more unexpected result would be formulating a
new heuristic for participation, or understanding a new domain
construct.  But it is just this sort of higher-level learning from
experience that we would like our systems to be able to support.

Further guidelines for structuring and participating in traditional
writers workshops are presented by Linda Elkin in
\cite[pp. 201--203]{gabriel2002writer}.  It is not at all clear that
the same ground rules should apply to computer systems.  For example,
one of Elkin's rules is that ``Quips, jokes, or sarcastic comments,
even if kindly meant, are inappropriate.''  Rather than forbidding
humour, it may be better for individual comments to be rated as
helpful or non-helpful.  Again, in the first instance, usefulness
and interest might be judged in terms of explicit criteria for serendipity;
see \cite{corneli15cc,pease2013discussion}.

\begin{table}[p]
\begin{tabular}{p{.95\columnwidth}}
{\bf\emph{Successful error}}  \\[.1cm]
\emph{example}:  Post-it\texttrademark\ notes \\[.1cm]
{\bf context:} Systems share interesting ideas even if they don't know directly how they will be useful.  \\
{\bf problem:} Systems should \emph{listen} with interest, too. \\
{\bf solution:} Listening systems can become more invested by remixing what they hear.\\
{\bf rationale:} If meaning follows usage, then using material in different ways can help to make meaning. Building robust meaning can take a long time. \\
{\bf resolution:} This heuristic can remind us to document what doesn't work and why, as well as what does.\\
\end{tabular}
\medskip

\begin{tabular}{p{.95\columnwidth}}
{\bf\emph{Side effect}}  \\[.1cm]
\emph{example}:  Nicotinamide used to treat side-effects of radiation therapy proves efficacious against tuberculosis. \\[.2cm]
{\bf context:} Presentations are experiments, and we don't know what the result will be. \\
{\bf problem:} Listeners need to be able to convey the way they are affected. \\
{\bf solution:} Feedback routines should report on the listener's ``sensory experience.''  This then needs to be integrated by the presenter.\\
{\bf rationale:} Different participants will have different ways of seeing things.\\
{\bf resolution:} Using this heuristic requires presenters to form and test hypotheses, and requires participants to communicate effectively about internal states. \\
\end{tabular}
\medskip

\begin{tabular}{p{.95\columnwidth}}
{\bf\emph{Wrong hypothesis}}  \\[.1cm]
\emph{example}:  Lithium, used in a control study, had an unexpected calming effect. \\[.1cm]
{\bf context:} If presentations are experiments, they should have a control. \\
{\bf problem:} But all listeners are ``guinea pigs''.\\
{\bf solution:} The presenter should make a model of the text that is being presented and note the parts that are presumed to be ``active ingredients.'' \\
{\bf rationale:} This will help focus on the most interesting and problematic parts of the text and conversation.\\
{\bf resolution:} This heuristic clarifies the nature of the experiment, and what may constitute surprising feedback. \\
\end{tabular}
\medskip

\begin{tabular}{p{.95\columnwidth}}
{\bf\emph{Outsider}}  \\[.1cm]
\emph{example}:  A mother suggests a new hypothesis to a doctor. \\[.1cm]
{\bf context:} The presenter aims learn from the critics, and the critics are here to help. \\
{\bf problem:} Can the critics learn as well?\\
{\bf solution:} Critics can draw on ``reverse feedback'' in the form of {\tt questions} asked by the presenter.\\
{\bf rationale:} This can help critics learn how to give clearer feedback.\\
{\bf resolution:} This heuristic points to the relevance of building models of \emph{others'} mental models.\\
\end{tabular}
\vspace{.1cm}
\caption{Reinterpreting van Andel's patterns of serendipity as heuristics to use in a computational workshop\label{tab:reinterpret}}
\end{table}

\bigskip

\begin{enumerate}[start=4]
\item \textbf{Discussion of how this would work more generally in
  computational creativity and perhaps in AI more generally with an
  eye toward producing effects like ``serendipity'' and
  ``emergence.''}
\begin{enumerate}
\item Section \item Feedback is the fundamental concept in \emph{cybernetics}.  \dec{Definition?}
\item ``Artificial and natural intelligence are both search'' -- you learn a few concepts the hard way and then use them to understand things you only have one shot at (Metaphors we live by).  Neo-diffusionist hypothesis: culture is like any other feature, if it helps you, you're more likely to keep it.  (Kashima).  Deacon's ``Theory of semantics''.  Human semantics can be replicated by statistical learning on large corpora.  Finch 1993, Bilovich and Bryson 2008.  The 75 second most frequent words give a good match to human semantics.  McDonald and Lowe, check the cosines.  Implicit University of Bath, Macfarlane.  Bryson 2008, Theory of Semantics.  ``What are the things that need to be preserved and what are the things that we can mutate?''  Primates not only remember how people treat themselves, they also remember how they treat each other (Second Order representation).  Evolvability is something that AI reseachers should be getting their hands on.  Dual replicators: culture and biology both evolve.  Within the genome, there are hierarchical representations and genes to flag ``zones of innovation'' (Richard Watson, machine learning, genetics, and evolvability)  Cross-over lets you stay close to the fitness peak (Yifei Wang).  \v{C}a\v{c}e and Bryson 2007, agent based modelling of communication costs, in Emergence and Evolution of Linguistic Communication.  ``What is transmitted is the replicator, but the unit of selection is the vehicle/interactor.''  Simpson's paradox.  Daniel Taylor, Evolution of the Social Contract -- everything is semi-private, disseration.  Lots of ways to create modular systems.  Jekaterina Novikova, transparently synthetic emotions for collaboration.  Swen Gaudl, learning from observation of human players and then coming up with better versions of the game using genetic programming.  London Futurists.  Human language and our values come out of our experiences and our \emph{sociality}.  E.g. punishing someone for being antisocial is contributing to your own good.  ``Chimps are moral patients.''
\begin{itemize}
\item (Clark 2013) quotes Ashby.  Homeostatics\ldots Stuff from Anil's talk.
  Wiener 1964 cites Ashby.  Conant and Ashby: ``Every good regulator
  of a system must be a model of that system.''  This leads to the
  ``free energy principle.''  Keep the organism within states in which
  it will receive the kinds of input it expects.
\item E.g.~emotional states are perception of physiological state but ``cognitive context'' matters.  (Seth 2013)
\end{itemize}
\item Feedback about feedback (\&c for higher orders) is relevant to thinking about \emph{learning} and \emph{communication}
\item \emph{Creativity} is often envisaged as cyclical process (e.g.~Dickie's
  art circle, Colton et al.~Iterative
  Development Expression Appreciation).  There are opportunities for
  embedded feedback at each step, and the process itself is ``akin
  to'' a feedback loop.
\begin{itemize}
\item A very intuitive point: customer feedback is a big thing, and increasingly present in online commerce and the social web (think e.g.~of Yelp).  The particular challenge here is to have the resources in order to effectively respond to the feedback.
\item Immunity (and autoimmunity) are examples of biological feedback systems.  Cf.~the Human Interaction Ontology.
\item Another philosophical point (maybe we can bring it in earlier)
  is that in the Writers Workshop model, feedback is somehow an
  economic ``externality.''  That is, the author gets the feedback
  ``for free.''  Other applications of the idea would be similar.
  This kicks off a number of related questions.
\item The first question is about ``how integrated'' the systems
  should be.  Are we willing to call an action taken by the system
  itself `feedback'?  Probably not: for that appelation we would
  require interaction with the world, causally determined but in some
  way stochastic effects.
\item This is related to Bergson's discussion of \emph{reflex} and
  (especially) Coase's discussion of firms.  In a more AI setting,
  this relates to the idea of \emph{sensors} and \emph{effectors}.  In
  dialogue models there are ideas of ``deontic scoreboards''
  (Brandom, Walton).
\item Different traditions in cognitive science: predictive processing
  e.g.~von Helmholtz.  The brain has no direct access.  There are
  uncertain signals about actual things in the world; so the brain
  needs to use inference.  (Clark 2013 BBS) The beholder's share.
  This maps onto ideas of predictive processing in cortical networks.
  Top down convey predictions, bottom up convey prediction errors
  (French 2009 Trends in Cognitive Science).  In the end, there's a
  sensible explanation.  We usually think that bottom up conveys the
  signal; but this outside-in pathway is built on feedback.  How
  strongly should we weight prediction errors versus predictions?
%% Sellars: reliable differential capacities to respond to environmental stimuli
\end{itemize}
\end{enumerate}
\end{enumerate}
