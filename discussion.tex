\section{Discussion}

\begin{itemize} 
\item To better evaluate how the Writers Workshop model helps us
  advance in our goal of incorporating feedback into artificial
  creativity, we can consider the model in terms of how it fits into
  related work. In particular, serendipity is a key concept within
  creativity, and AI more generally, which the Writers Workshop could
  assist computational progress.
\item Each contributing system should come to the workshop with at
  least a basic awareness of the protocol, with work to share, and
  prepared to give constructive feedback to other systems.  The
  workshop itself needs to be prepared, with a suitable communication
  platform and a moderator.  In order to get value out of the
  experience, systems (and their wranglers) should ideally have
  questions they are investigating.  Systems should be prepared to
  give feedback, and to carry out evaluations of the helpfulness (or
  not) of feedback from other systems and of the experience overall.
  Developing systems that could successfully navigate this
  collaborative exercise would be a significant advance in the field
  of computational creativity.  Since the experience is about learning
  rather than winning, there is little motivation to ``game the
  system'' (cf. \citeNP{lenat1983eurisko}).
\end{itemize}
