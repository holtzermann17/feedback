\section{Concluding discussion and future directions}

 We have described a \emph{general} and \emph{computationally feasible}  
 model for using feedback in AI systems, particularly creative systems. 
 The Writers Workshop concept, borrowed from creative writing, is transformed into a model of a 
 structured approach to eliciting,  processing and learning from feedback. 
To better evaluate how the Writers Workshop model helps us
  advance in our goal of incorporating feedback into artificial
  creativity, we critically considered how the model fits into
  related work. In particular, we found that serendipity, a key concept within
  creativity and AI more generally, is a concept with which the Writers Workshop model could
  assist computational progress.

In practical terms, our pilot implementation work (Section \ref{sec:implementation}) shows that the model transfers across to computational implementation. The practical work highlights various considerations to be acknowledged when
 developing systems using the Writers Workshop model:
 [***JOE, ANYTHING TO ADD AS A RESULT OF THE PRACTICAL WORK SO FAR IMPLEMENTING THE MODEL? **]

\begin{itemize}
\item Each contributing system should come to the workshop with at
  least a basic awareness of the workshop protocol, with work to share, and
  prepared to give constructive feedback to other systems.  
  \item The
  workshop itself needs to be prepared, with a suitable communication
  platform and a moderator.  
 \item  A controlled vocabulary for communications and 
  interaction would be a worthwhile pursuit of future research, perhaps based on
  an ontology such as the Interaction Network Ontology.\footnote{The Interaction Network Ontology is documented at http://www.ontobee.org/browser/index.php?o=INO . Its URI is http://svn.code.sf.net/p/ino/code/trunk/src/ontology/INO.owl .}
  \item   To get value out of the
  experience, systems (and their wranglers) should ideally have
  questions they are investigating.  
  \item Systems should be prepared to
  give feedback, and to carry out evaluations of the helpfulness (or
  not) of feedback from other systems and of the experience overall.
  \end{itemize}
  
  Developing systems that could successfully navigate this
  collaborative exercise would be a significant advance in the field
  of computational creativity.  Since the experience is about learning
  rather than winning, there is little motivation to ``game the
  system'' (cf. \citeNP{lenat1983eurisko}). Instead the emphasis is squarely upon 
  mutual benefit: computational systems
  helping to develop each other through communication and feedback.