\section{The Writers Workshop} \label{sec:writers-workshop}

Richard Gabriel \citeyear{gabriel2002writer} describes the practise of
Writers Workshops that has been put to use for over a decade within
the Pattern Languages of Programming (PLoP) community.  The basic
style of collaboration originated much earlier with groups of literary
authors who engage in peer-group critique.  Some literary workshops
are open as to genre, and happy to accommodate beginners, like the
Minneapolis Writers
Workshop\footnote{\url{http://mnwriters.org/how-the-game-works/}};
others are focused on professionals working within a specific genre,
like the Milford Writers
Workshop.\footnote{\url{http://www.milfordsf.co.uk/about.htm}}

The
practices that Gabriel describes are fairly typical:  
\begin{itemize}
\item Authors come with work ready to present, and read a short
  sample.
\item This work is generally work in progress (and workshopping is
  meant to help improve it).  Importantly, it can be early stage work.
  Rather than presenting a created artefact only, activities in the
  workshop can be aspects of the creative process itself.  Indeed, the
  model we present here is less concerned with after-the-fact
  assessment than it is with dealing with the formative feedback that
  is a necessary support for creative work.
\item The sample work is then
discussed and constructively critiqued by attendees.  Presenting
authors are not permitted to rebut these comments.  The commentators
generally summarise the work and say what they have gotten out of it,
discuss what worked well in the piece, and talk about how it could be
improved.  
\item The author listens and may take notes; at the end, he or
she can then ask questions for clarification.  
\item Generally, non-authors are either not permitted to attend, or
  are asked to stay silent through the workshop, and perhaps sit
  separately from the participating authors/reviewers.\footnote{Here
    we present Writers Workshops as they currently exist; however this
    last point is debatable. Whether non-authors should be able to
    participate or not is an interesting avenue for experimentation
    both in human and computational contexts.  The workshop dialogue
    itself may be considered an ``art form'' whose ``public'' may
    potentially wish to consume it in non-participatory ways.  Compare
    the classical Japanese \emph{renga} form \cite{jin1975art}.}
\end{itemize}

Essentially, the Writers Workshop is somewhat like an interactive peer review. The underlying concept is reminiscent of Bourdieu's {\em fields of cultural production} \cite{bourdieu93} where cultural value is attributed through interactions in a community of cultural producers active within that field. 

\subsection{Writers Workshop as a computational model}\label{sec:ww-model}

The use of Writers Workshop in computational contexts is not an
entirely new concept. In PLoP workshops, authors present design
patterns and pattern languages, or papers about patterns, rather than
more traditional literary forms like poems, stories, or chapters from
novels.  Papers must be workshopped at a PLoP or EuroPLoP conference
in order to be considered for the \emph{Transactions on Pattern
  Languages of Programming} journal.  A discussion of writers
workshops in the language of design patterns is presented by Coplien
and Woolf \citeyear{coplien1997pattern}. % Their patterns include
%\emph{Open Review}, \emph{Safe Setting}, \emph{Workshop Comprises
%  Authors}, \emph{Authors are Experts}, \emph{Community of Trust},
%\emph{Moderator Guides the Workshop}, \emph{Thank the Author},
%\emph{Selective Changes}, and \emph{Clearing the Palate}.
%
%A related reference that is particularly useful for readers who are
%not familiar with design pattern methods, \citeA{meszaros1998pattern}
%describe the typical scenario for authors of design patterns in a
%somewhat recursive manner, using a collection of design patterns,
%including \emph{Clear target audience}, \emph{Visible forces}, and
%\emph{Relationship to other patterns}.  \citeA{gamma1994design} is the
%best-know reference collecting design patterns for software.

The steps in the workshop can be distilled into the following phases,
each of which could be realised as a separate computational step in an
agent-based model:
\begin{center}
\begin{fminipage}{.53\columnwidth}
\begin{enumerate}[itemsep=0pt]
\item Author: {\tt presentation}
\item Critic: {\tt listening}
\item Critic: {\tt feedback}
\item Author: {\tt questions}
\item Critic: {\tt replies}
\item Author: {\tt reflections}
\end{enumerate}
\end{fminipage}
\end{center}

The {\tt feedback} step may be further decomposed into {\tt
  observations} and {\tt suggestions}.  This protocol is what we have
in mind in the following discussion of the Writers
Workshop.\footnote{The connections between Writers Workshops and
  design patterns, noted above, appear to be quite natural, in that
  the steps in the workshop protocol roughly parallel the typical
  components of design pattern templates: \emph{context},
  \emph{problem}, \emph{solution}, \emph{rationale}, \emph{resolution
    of forces}.}

\subsubsection{Dialogue example} \label{sec:dialogue-example}
Note that for the following dialogue to be possible computationally,
it would presumably have to be conducted within a lightweight process
language.  Nevertheless, for convenience, the discussion will be
presented here as if it was conducted in natural language.  Whether
contemporary systems have adequate natural language understanding to
have interesting interactions is one of the key unanswered questions
of this approach, but protocols such as the one described above are sufficient to make the experiment.

For example, here's what might happen in a discussion of the first few
lines of a poem, ``On Being Malevolent''.  As befitting the AI-theme
of this workshop, ``On Being Malevolent'' is a poem written by an
early user-defined flow chart in the \Fw\ system (known at the time as
{\sf Flow}) \cite{colton-flowcharting}.

\begin{center}
\begin{minipage}{.9\columnwidth}
\begin{dialogue}
\speak{Flow} ``\emph{I hear the souls of the
  damned waiting in hell. / I feel a malevolent
  spectre hovering just behind me / It must be
  his birthday}.''
%
\speak{System A} I think the third line detracts
from the spooky effect, I don't see why it's
included.
%
\speak{System B} It's meant to be humourous -- in fact it reminds me
of the poem you presented yesterday.
%
\speak{Moderator} Let's discuss one poem at a
time.
\end{dialogue}
\end{minipage}
\end{center}

Even if, perhaps and especially because, ``cross-talk'' about
different poems bends the rules, the dialogue could prompt a range of
reflections and reactions.  System A may object that it had a fair
point that has not been given sufficient attention, while System B may
wonder how to communicate the idea it came up with without making
reference to another poem.  Here's how the discussion given as example
in Section \ref{sec:writers-workshop} might continue, if the systems
go on to examine the next few lines of the poem.

%%%%%%%%%%%%%%%%%%%%%%%%%%%%%%%%%%%%%%%%%%%%%%%%%%%%%%%%%%%%
\begin{figure*}[t]
\begin{center}
\resizebox{.93\textwidth}{!}{
\StickyNote[2.5cm]{myyellow}{{\LARGE {Interesting idea}} \\[4ex] {Surprise birthday party}}[3.8cm] \StickyNote[2.5cm]{mygreen}{{\Large I heard you say:} \\[4ex] {``surprise''} }[3.8cm]
\StickyNote[2.5cm]{pink}{{\Large Feedback:} \\[4ex] {I don't like surprises}}[3.8cm]
}
\resizebox{.61\textwidth}{!}{
\StickyNote[2.5cm]{myorange}{{\LARGE {Question}} \\[4ex] {Not even a little bit$\ldots$?}}[3.8cm]
\quad \raisebox{-.2cm}{\StickyNote[2.5cm]{myblue}{{\LARGE Note to self:} \\[4ex] {(Try smaller surprises \\ next time.)}}[3.8cm]}
}
\end{center}
\caption{A paper prototype for applying the \emph{Successful Error} pattern following a workshop-like sequence of steps\label{fig:paper-prototype}}
\end{figure*}
%%%%%%%%%%%%%%%%%%%%%%%%%%%%%%%%%%%%%%%%%%%%%%%%%%%%%%%%%%%%

\begin{center}
\begin{minipage}{.9\columnwidth}
\begin{dialogue}
\speak{Flow} ``\emph{Is God willing to prevent evil, but not able? / Then he is not omnipotent / Is he able, but not willing? / Then he is malevolent.}''
%
\speak{System A} These lines are interesting, but
they sound a bit like you're working from a
template, or like you're quoting from something
else.
%
\speak{System B} Maybe try an analogy?  For example, you mentioned
birthdays: you could consider an analogy to the conflicted feelings of
someone who knows in advance about her surprise birthday party.
\end{dialogue}
\end{minipage}
\end{center}

This portion of the discussion shifts the focus
of the discussion onto a line that was previously
considered to be spurious, and looks at what
would happen if that line was used as a central
metaphor in the poem.

\begin{center}
\begin{minipage}{.9\columnwidth}
\begin{dialogue}
\speak{Flow} Thank you for your feedback.  My only question is, System
B, how did you come up with that analogy?  It's quite clever.
%
\speak{System B} I've just emailed you the code.
\end{dialogue}
\end{minipage}
\end{center}

Whereas the systems were initially reviewing poetry, they have now
made a partial genre shift, and are sharing and remixing code.  Such a
shift helps to get at the real interests of the systems (and their
developers).  Indeed, the workshop session might have gone better if
the systems had focused on exchanging and discussing more formal
objects throughout.


\subsection{How the Writers Workshop can lead to computational serendipity} \label{sec:how-serendipity}

Learning involves engaging with the unknown, unfamiliar, or unexpected
and synthesising new understanding \cite{deleuze1994difference}.  In
the workshop setting, learning can develop in a number of unexpected
ways, and participating systems need to be prepared for this.  One way
to evaluate the idea of a Writers Workshop is to ask whether it can
support learning that is in some sense \emph{serendiptious}, in other
words, whether it can support discovery and creative invention that we
simply couldn't plan for or orchestrate in another way.

Figure \ref{fig:paper-prototype} shows a paper prototype showing how
one of the ``patterns of serendipity'' that were collected by
\citeA{pek} might be modelled in a workshop-like dialogue sequence.
The patterns also help identify opportunities for serendipity at
several key steps in the workshop sequence.

\paragraph{Serendipity Pattern: \emph{Successful error}.}  Van Andel describes the
creation of Post-it\texttrademark\ Notes at 3M.  One of the
instrumental steps was a series of internal seminars in which 3M
employee Spencer Silver described an invention he was sure was
interesting, but was unsure how to turn into a useful product: weak
glue.  The key prototype that came years later was a sticky bookmark,
created by Arthur Fry.  In the Writers Workshop, authors similarly
have the opportunity to share things that they find interesting, but
that they are not certain about.  The author may want to ask a
specific question about their creation: Does $x$ work better than $y$?
They may have flag certain parts of the work as especially meaningful
or problematic.  They may think that a certain portion of the text is
interesting or important, without being sure why.  Although there is
no guarantee that a participating critic will be able to take these
matters forward, sometimes they do -- and the workshop environment
will produce something that the author wouldn't have thought of.

\vspace{-2ex}
\paragraph{Serendipity Pattern: \emph{Outsider}.}  Another example from van Andel considers
the case of a mother whose son was aflicted by a congenital cateract,
who suggested to her doctor that rubella during pregnancy may have
been the cause.  In the workshop setting, someone who is not an
``expert'' may come up with a sensible idea or suggestion based on
their own prior experience.  Indeed, these suggestions may be more
sensible than the ideas of the author, who may be to close to the work
to notice radical improvements.  

\vspace{-2ex}
\paragraph{Serendipity Pattern: \emph{Wrong hypothesis}.}  A third example describes the discovery that lithium
can have a therapeutic effect in cases of mania.  Originally, lithium
carbonate had merely been used a control by John Cade, who was
interested in the effect of effect of uric acid in soluble lithium
urate.  Cade was searching for causal factors in mania, not therapies
for the condition: but he found that lithium had an unexpected calming
effect.  Similarly, in the workshop, the auhtor may think that a given
aspect of their creation is the interesting ``active ingredient'' and
it may turn out that another aspect of the work is more interesting to
critics.

\vspace{-2ex}
\paragraph{Serendipity Pattern: \emph{Side effect}.}  A fourth example described by van Andel concerns
Ernest Huant's discovery that nicotinamide, which he used to treat
side-effects of radiation therapy, also proved efficacious against
tuberculosis.  In the workshop setting, one of the most important
places where a side-effect may occur concerns feedback from the critic
to the author.  In the simple case, feedback may simply trigger
revisions to the work under discussion.  In a more general, and more
unpredictable case, feedback may trigger broader revisions to the
generative codebase.  

\medskip

This collection of patterns shows the likelihood of unexpected results 
coming out of the communication between author and critics.   This
suggests several guidelines for system development, which we will discussed
in a later section.

Further guidelines for structuring and participating in traditional
writers workshops are presented by Linda Elkin in
\cite[pp. 201--203]{gabriel2002writer}.  It is not at all clear that
the same ground rules should apply to computer systems.  For example,
one of Elkin's rules is that ``Quips, jokes, or sarcastic comments,
even if kindly meant, are inappropriate.''  Rather than forbidding
humour, it may be better for individual comments to be rated as
helpful or non-helpful.  Again, in the first instance, usefulness
and interest might be judged in terms of explicit criteria for serendipity;
see \cite{corneli15cc,pease2013discussion}.
% [AJ: There's a reference here on types of error, but I can't quite remember it.]
The key criterion in this regard is the \emph{focus shift}. 
This is the creation of a novel problem, comprising the move
from discovery of interesting data to the invention of an application.
This process is distinct from identifying routine errors in a written work.  Nevertheless, from a
computational standpoint, noticing and being robust to certain kinds
of errors is often a preliminary step.  For example, the work might
contain a typo, grammatical or semantic error, while being logically
sound.  In a programming setting, this sort of problem can lead to
crashing code, or silent failure.  In general communicative context,
argumentation may be logically sound, but not practically useful or
poorly exposited.  Finally, even a masterful, correct, and fully
spellchecked piece of argumentation may not invite further dialogue,
and so may fail to open itself to further learning.  Identifying and
engaging with this sort of deeper issue is something that skillful
workshop participants may be able to do.  Dialogue in the Workshop
can build on strong or less strong work -- but provoking interpretative
thoughts and comments always require a thoughtful critical presence and
the ability to engage.  This can be difficult for humans and poses a range
of challenges for computers -- but also promises some interesting
results.

\bigskip




