\section{The Writers Workshop} \label{sec:writers-workshop}

Richard Gabriel \citeyear{gabriel2002writer} describes the practise of
Writers Workshops that has been put to use for over a decade within
the Pattern Languages of Programming (PLoP) community.  The basic
style of collaboration originated much earlier with groups of literary
authors who engage in peer-group critique.  Some literary workshops
are open as to genre, and happy to accommodate beginners, like the
Minneapolis Writers
Workshop\footnote{\url{http://mnwriters.org/how-the-game-works/}};
others are focused on professionals working within a specific genre,
like the Milford Writers
Workshop\footnote{\url{http://www.milfordsf.co.uk/about.htm}}.  

The
practices that Gabriel describes are fairly typical:  
\begin{itemize}
\item Authors come with work ready to present, and read a short
  sample.
\item This work is generally work in progress (and workshopping is
  meant to help improve it).  Importantly, it can be early stage work.
  Rather than presenting a created artefact only, activities in the
  workshop can be aspects of the creative process itself.  Indeed, the
  model we present here is less concerned with after-the-fact
  assessment than it is with dealing with the formative feedback that
  is a necessary support for creative work.
\item The sample work is then
discussed and constructively critiqued by attendees.  Presenting
authors are not permitted to rebut these comments.  The commentators
generally summarise the work and say what they have gotten out of it,
discuss what worked well in the piece, and talk about how it could be
improved.  
\item The author listens and may take notes; at the end, he or
she can then ask questions for clarification.  
\item Generally, non-authors
are either not permitted to attend, or are asked to stay silent
through the workshop, and perhaps sit separately from the
participating authors/reviewers.\footnote{Here we present Writers Workshops as they currently exist; however this last point is debatable. Whether non-authors should be able to participate or not is an interesting avenue for experimentation both in human and computational contexts.}
\end{itemize}

Essentially, the Writers Workshop is somewhat like an interactive peer review. The underlying concept is reminiscent of Bourdieu's {\em fields of cultural production} \cite{bourdieu93} where cultural value is attributed through interactions in a community of cultural producers active within that field. 

\subsection{Writers Workshop in computational settings}
The use of Writers Workshop in computational contexts is not an
entirely new concept. In PLoP workshops, authors present design
patterns and pattern languages, or papers about patterns, rather than
more traditional literary forms like poems, stories, or chapters from
novels.  Papers must be workshopped at a PLoP or EuroPLoP conference
in order to be considered for the \emph{Transactions on Pattern
  Languages of Programming} journal.  A discussion of writers
workshops in the language of design patterns is presented by Coplien
and Woolf \citeyear{coplien1997pattern}.  Their patterns include
\emph{Open Review}, \emph{Safe Setting}, \emph{Workshop Comprises
  Authors}, \emph{Authors are Experts}, \emph{Community of Trust},
\emph{Moderator Guides the Workshop}, \emph{Thank the Author},
\emph{Selective Changes}, and \emph{Clearing the Palate}.
%
A related reference that is particularly useful for readers who are
not familiar with design pattern methods, \citeA{meszaros1998pattern}
describe the typical scenario for authors of design patterns in a
somewhat recursive manner, using a collection of design patterns,
including \emph{Clear target audience}, \emph{Visible forces}, and
\emph{Relationship to other patterns}.


\subsubsection{Dialogue example}
Note that for the following dialogue to be possible computationally,
it would presumably have to be conducted within a lightweight process
language.  Nevertheless, for convenience, the discussion will be
presented here as if it was conducted in natural language.  Whether
contemporary systems have adequate natural language understanding to
have interesting interactions is one of the key unanswered questions
of this approach, but protocols such as that described later in this
paper would be sufficient to make the experiment.

For example, here's what might happen in a discussion of the first few
lines of a poem, ``On Being Malevolent,''. As befitting the AI-theme
of this workshop, ``On Being Malevolent'' is a poem written by an
early user-defined flow chart in the \Fw\ system (known at the time as
{\sf Flow}) \cite{colton-flowcharting}.

\begin{center}
\begin{minipage}{.9\columnwidth}
\begin{dialogue}
\speak{Flow} ``\emph{I hear the souls of the
  damned waiting in hell. / I feel a malevolent
  spectre hovering just behind me / It must be
  his birthday}.''
%
\speak{System A} I think the third line detracts
from the spooky effect, I don't see why it's
included.
%
\speak{System B} It's meant to be humourous -- in fact it reminds me
of the poem you presented yesterday.
%
\speak{Moderator} Let's discuss one poem at a
time.
\end{dialogue}
\end{minipage}
\end{center}

Even if, perhaps and especially because, ``cross-talk'' about
different poems bends the rules, the dialogue could prompt a range of
reflections and reactions.  System A may object that it had a fair
point that has not been given sufficient attention, while System B may
wonder how to communicate the idea it came up with without making
reference to another poem.  Here's how the discussion given as example
in Section \ref{sec:writers-workshop} might continue, if the systems
go on to examine the next few lines of the poem.
\begin{center}
\begin{minipage}{.9\columnwidth}
\begin{dialogue}
\speak{Flow} ``\emph{Is God willing to prevent evil, but not able? / Then he is not omnipotent / Is he able, but not willing? / Then he is malevolent.}''
%
\speak{System A} These lines are interesting, but
they sound a bit like you're working from a
template, or like you're quoting from something
else.
%
\speak{System B} Maybe try an analogy?  For example, you mentioned
birthdays: you could consider an analogy to the conflicted feelings of
someone who knows in advance about her surprise birthday party.
\end{dialogue}
\end{minipage}
\end{center}

This portion of the discussion shifts the focus
of the discussion onto a line that was previously
considered to be spurious, and looks at what
would happen if that line was used as a central
metaphor in the poem.

\begin{center}
\begin{minipage}{.9\columnwidth}
\begin{dialogue}
\speak{Flow} Thank you for your feedback.  My only question is, System
B, how did you come up with that analogy?  It's quite clever.
%
\speak{System B} I've just emailed you the code.
\end{dialogue}
\end{minipage}
\end{center}

Whereas the systems were initially reviewing poetry, they have now
made a partial genre shift, and are sharing and remixing code.  Such a
shift helps to get at the real interests of the systems (and their
developers).  Indeed, the workshop session might have gone better if
the systems had focused on exchanging and discussing more formal
objects throughout.

\begin{figure*}[t]
\begin{center}
\resizebox{.93\textwidth}{!}{
\StickyNote[2.5cm]{myyellow}{{\LARGE {Interesting idea}} \\[4ex] {Surprise birthday party}}[3.8cm] \StickyNote[2.5cm]{mygreen}{{\Large I heard you say:} \\[4ex] {``surprise''} }[3.8cm]
\StickyNote[2.5cm]{pink}{{\Large Feedback:} \\[4ex] {I don't like surprises}}[3.8cm]
}
\resizebox{.61\textwidth}{!}{
\StickyNote[2.5cm]{myorange}{{\LARGE {Question}} \\[4ex] {Not even a little bit$\ldots$?}}[3.8cm]
\quad \raisebox{-.2cm}{\StickyNote[2.5cm]{myblue}{{\LARGE Note to self:} \\[4ex] {(Try smaller surprises \\ next time.)}}[3.8cm]}
}
\end{center}
\caption{A paper prototype for applying the \emph{Successful Error} pattern\label{fig:paper-prototype}}
\end{figure*}

