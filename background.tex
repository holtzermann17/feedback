\section{The Writers Workshop} \label{sec:writers-workshop}

Richard Gabriel \citeyear{gabriel2002writer} describes the practise of
Writers Workshops that has been put to use for over a decade within
the Pattern Languages of Programming (PLoP) community.  The basic
style of collaboration originated much earlier with groups of literary
authors who engage in peer-group critique.  Some literary workshops
are open as to genre, and happy to accommodate beginners, like the
Minneapolis Writers
Workshop\footnote{\url{http://mnwriters.org/how-the-game-works/}};
others are focused on professionals working within a specific genre,
like the Milford Writers
Workshop\footnote{\url{http://www.milfordsf.co.uk/about.htm}}.  

The
practices that Gabriel describes are fairly typical:  
\begin{itemize}
\item Authors come with work ready to present, and read a short
  sample.
\item This work is generally work in progress (and workshopping is
  meant to help improve it).  Importantly, it can be early stage work.
  Rather than presenting a created artefact only, activities in the
  workshop can be aspects of the creative process itself.  Indeed, the
  model we present here is less concerned with after-the-fact
  assessment than it is with dealing with the formative feedback that
  is a necessary support for creative work.
\item The sample work is then
discussed and constructively critiqued by attendees.  Presenting
authors are not permitted to rebut these comments.  The commentators
generally summarise the work and say what they have gotten out of it,
discuss what worked well in the piece, and talk about how it could be
improved.  
\item The author listens and may take notes; at the end, he or
she can then ask questions for clarification.  
\item Generally, non-authors
are either not permitted to attend, or are asked to stay silent
through the workshop, and perhaps sit separately from the
participating authors/reviewers. {\em This is somewhat akin to the ideas about fields of cultural production \cite{bourdieu93}, where value is perceived as coming from interactions between a peer community of cultural producers i.e. specifically those who are active within that field. *** AJ Check I'm not talking rubbish? ***}\footnote{Here we present Writers Workshops as they currently exist; however this last point is debatable. Whether non-authors should be able to participate or not is an interesting avenue for experimentation both in human and computational contexts.}
\end{itemize}

There are similarities between the
Writers Workshops and classical practices of group composition
\cite{jin1975art} and dialectic \cite{dialectique}, and the workshop
may be considered an artistic or creative space in its own right.

For example, here's what might happen in a discussion of the first few lines of
``On Being Malevolent,''. As befitting the AI-theme of this workshop, ``On Being Malevolent'' is a poem written by an early user-defined flow chart
in the \Fw\ system (known at the time as {\sf Flow})
\cite{colton-flowcharting}.\footnote{Note that for this dialogue to be
possible computationally, it would presumably have to be conducted within a
lightweight process language, as discussed above.  Nevertheless, for
convenience, the discussion will be presented here as if it was
conducted in natural language.  Whether contemporary systems have
adequate natural language understanding to have interesting
interactions is one of the key unanswered questions of this approach,
but protocols such as that described later in this paper would be sufficient to
make the experiment.}

\begin{center}
\begin{minipage}{.9\columnwidth}
\begin{dialogue}
\speak{Flow} ``\emph{I hear the souls of the
  damned waiting in hell. / I feel a malevolent
  spectre hovering just behind me / It must be
  his birthday}.''
%
\speak{System A} I think the third line detracts
from the spooky effect, I don't see why it's
included.
%
\speak{System B} It's meant to be humourous -- in fact it reminds me
of the poem you presented yesterday.
%
\speak{Moderator} Let's discuss one poem at a
time.
\end{dialogue}
\end{minipage}
\end{center}




The use of Writers Workshop in computational contexts is not an entirely new concept. In PLoP workshops, authors present design patterns and pattern
languages, or papers about patterns, rather than more traditional
literary forms like poems, stories, or chapters from novels.  Papers
must be workshopped at a PLoP or EuroPLoP conference in order to be
considered for the \emph{Transactions on Pattern Languages of
  Programming} journal.  A discussion of writers workshops
in the language of design patterns is presented by
Coplien and Woolf \cite{coplien1997pattern}.  Their patterns include:
\begin{center}
{\small
\begin{tabular}{l@{\hspace{.2cm}}l@{\hspace{.2cm}}l}
\emph{Open Review} & \emph{Safe Setting} & \emph{Workshop Comprises Authors} \\
\emph{Authors are Experts} & \emph{Community of Trust} & \emph{Moderator Guides the Workshop} \\
\emph{Thank the Author} & \emph{Selective Changes} & \emph{Clearing the Palate} \\
\end{tabular}
}
\end{center}


\begin{table}[p]
\begin{tabular}{p{.95\columnwidth}}
{\bf\emph{Successful error}}  \\
\emph{Van Andel's example}:  Post-it\texttrademark\ notes \\[.2cm]
{\tt presentation} Systems should be prepared to share interesting ideas even if they don't know directly how they will be useful.  \\
{\tt listening}    Systems should listen with interest, too. \\
{\tt feedback}     Even interesting ideas may not be ``marketable.''\\
{\tt questions}    How is your suggestion useful? \\
{\tt reflections}  New combinations of ideas take a long time to realise, and many different ideas may need to be combined in order to come up with something useful.\\
\end{tabular}
\medskip

\begin{tabular}{p{.95\columnwidth}}
{\bf\emph{Side effect}}  \\
\emph{Van Andel's example}:  Nicotinamide used to treat side-effects of radiation therapy proves efficacious against tuberculosis. \\[.2cm]
{\tt presentation} Systems should use their presentation as an experiment. \\
{\tt listening}    Listeners should allow themselves to be affected by what they are hearing. \\
{\tt feedback}     Feedback should convey the nature of the effect.\\
{\tt questions}    The presenter may need to ask follow-up questions to gain insight. \\
{\tt reflections}  Form a new hypothesis before seeking a new audience. \\
\end{tabular}
\medskip

\begin{tabular}{p{.95\columnwidth}}
{\bf\emph{Wrong hypothesis}}  \\
\emph{Van Andel's example}:  Lithium, used in a control study, had an unexpected calming effect. \\[.2cm]
{\tt presentation} How is this presentation interpretable as a (``natural'') control study? \\
{\tt listening}    Listeners are ``guinea pigs''.\\
{\tt feedback}     Discuss side-effects that do not necessarily correspond to the author's perceived intent. \\
{\tt questions}    Zero in on the most interesting part of the conversation.\\
{\tt reflections}  Revise hypotheses to correspond to the most surprising feedback. \\
\end{tabular}
\medskip

\begin{tabular}{p{.95\columnwidth}}
{\bf\emph{Outsider}}  \\
\emph{Van Andel's example}:  A mother suggests a new hypothesis to a doctor. \\[.2cm]
{\tt presentation} The presenter is here to learn from the audience. \\
{\tt listening}   The audience is here to give help, but also to get help.\\
{\tt feedback}     Feedback will inevitably draw on previous experiences and ideas.\\
{\tt questions}    What is the basis for that remark?\\
{\tt reflections}  How can I implement the suggestions?\\
\end{tabular}
\vspace{.2cm}
\caption{Reinterpreting patterns of serendipity for use in a computational workshop\label{tab:reinterpret}}
\end{table}


\begin{figure*}[t]
\begin{center}
\resizebox{.93\textwidth}{!}{
\StickyNote[2.5cm]{myyellow}{{\LARGE {Interesting idea}} \\[4ex] {Surprise birthday party}}[3.8cm] \StickyNote[2.5cm]{mygreen}{{\Large I heard you say:} \\[4ex] {``surprise''} }[3.8cm]
\StickyNote[2.5cm]{pink}{{\Large Feedback:} \\[4ex] {I don't like surprises}}[3.8cm]
}
\resizebox{.61\textwidth}{!}{
\StickyNote[2.5cm]{myorange}{{\LARGE {Question}} \\[4ex] {Not even a little bit$\ldots$?}}[3.8cm]
\quad \raisebox{-.2cm}{\StickyNote[2.5cm]{myblue}{{\LARGE Note to self:} \\[4ex] {(Try smaller surprises \\ next time.)}}[3.8cm]}
}
\end{center}
\caption{A paper prototype for applying the \emph{Successful Error} pattern\label{fig:paper-prototype}}
\end{figure*}

\begin{enumerate}
\item \textbf{Here is the writers workshop idea -- to be merged with the ``old text'' above}
\item[] The steps:
\begin{enumerate}
\item A: {\tt presentation}
\item C: {\tt listening}
\item C: {\tt feedback} ({\tt observations}{\tt+}{\tt suggestions})
\item A: {\tt questions}
\item C: {\tt replies}
\item A: {\tt reflections}
\end{enumerate}
\item[] Previous applications:
\begin{enumerate}[label=(\roman*)]
\item We sketched a ``theory of poetics rooted in the making of
  boundary-crossing objects and processes''
\item We ``described a system that can (sometimes) make `highly
  serendipitous' creative advances in computer poetry'' while
  ``drawing attention to theoretical questions related to program
  design in an autonomous programming context.''
\end{enumerate}
\end{enumerate}
