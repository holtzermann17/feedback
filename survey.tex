\section{Computational Model of feedback based on writers workshop}\label{sec:ww-model}

\subsection{Protocol}
In order to facilitate this sort of interaction, it would be necessary
for systems to implement a basic protocol related to
%%
{\tt presentation}, {\tt listening}, {\tt
  feedback}, {\tt questions}, and {\tt
  reflections}.
%%

\begin{description}
\item[Presentation:] definition ***
\item[Listening:] definition ***
\item[Feedback:] definition ***
\item[Questions:] definition ***
\item[Reflections:] definition ***
\end{description}
This protocol guides participants through the different stages of the Writers Workshop.\footnote{This protocol could be thought of as a light-weight template for
creating design patterns that guide system-level participation in the
context specified by Coplien and Woolf's pattern language for writers
workshops.  ***REF }

\subsection{Communications and feedback}

We would need a neutral environment that is not hard to develop for:
the \Fw\ system described in Section \ref{sec:implementation}
offers one such possibility.  With this system, the basic operating
logic of the Workshop could be spelled out as a flowchart, and
contributing systems could use flowcharts as the basic medium for
sharing their presentations, feedback, and questions.  Developing
around a process language of this sort partially obviates the need for
participating systems to have strong natural language processing
capabilities.  
%
Post-it\texttrademark\ notes, which have provided us with a useful
example of serendipitous discovery, also provide indicative strategies
from the world of paper prototyping (Figure \ref{fig:paper-prototype}).

Gordon Pask's conversation theory, reviewed in
\cite{conversation-theory-review,boyd2004conversation}, goes
considerably beyond what we have presented here as a simple process
language, although there are structural parallels.  In a basic
Pask-style learning conversation: (0) Conversational participants are
carrying out some actions and observations; (1) naming and recording
what action is being done; (2) asking and explaining why it works the
way it does; (3) carrying out higher-order methodological discussion;
and (4) trying to figure out why unexpected results occured \cite[p. 190]{boyd2004conversation}.

Naturally, variations to the underlying system, protocol, and the
schedule of events should be considered depending on the needs and
interests of participants, and several variants can be tried.  On a
pragmatic basis, if the Workshop proved quite useful to participants,
it could be revised to run monthly, weekly, or
continuously.\footnote{For a comparison case in computer Go, see
  \url{http://cgos.computergo.org/}.}

\begin{enumerate}[start=2]
\item \textbf{Here we refine the idea and turn it into a general model
  that for incorporating feedback within computational models of the
  creative process.}
\item[] Intuitive examples of application areas
\begin{enumerate}
\item learning/training
%% Supervised learning,
%% Reinforcement learning (Temporal Difference Learning)
%% Evolution
\item guidance/control
\end{enumerate}
\item[] What can feedback be about in general terms? \dec{Survey}
\begin{enumerate}
\item Patterns that \emph{match} and any \emph{exceptions}. 
\begin{itemize}
\item Current computer programs are able to identify known patterns
  and ``close matches'' in data sets from certain domains, like music
  \cite{meredith2002algorithms}.  Identifying known patterns is a
  special case of the more general cocept of \emph{pattern mining}
  \cite{bergeron2007representation}.  In particular, the ability to
  extract \emph{new} higher order patterns that describe exceptions is
  an example of ``learning from feedback.''  Deep learning and
  evolutionary models increasingly use this sort of idea to facilitate
  strategic discovery \cite{samothrakis2011approximating}.  Similar
  ideas are considered in business application under the heading
  ``process mining'' \cite{van2011process}
\end{itemize}
\item Progress relative to explicit or adduced \emph{exploration}
  (knowledge and accuracy) or \emph{exploitation} (directional,
  task-based) goals.
\item \emph{Quantity}, \emph{Variety}, and \emph{Order} of produced
  objects or behaviours
\item New relationships among produced objects and behaviours drawing
  on a common field of reference
\end{enumerate}
\item[] How can feedback be understood and used? \dec{Survey}
\begin{enumerate}
\item Update knowledge base with new facts (accept statements,
  possibly with provenance)
\item Note similarity to \emph{iterative development}
\item Reflection: describe relationships between produced objects and
  behaviours and feedback
\item Reflection: Higher-order patterns e.g.~new patterns that
  describe the identified exceptions
\end{enumerate}
\end{enumerate}
