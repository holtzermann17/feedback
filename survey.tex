\begin{enumerate}[start=2]
\item \textbf{Here we refine the idea and turn it into a general model
  that for incorporating feedback within computational models of the
  creative process.}
\item[] Intuitive examples of application areas
\begin{enumerate}
\item learning/training
%% Supervised learning,
%% Reinforcement learning (Temporal Difference Learning)
%% Evolution
\item guidance/control
\end{enumerate}
\item[] What can feedback be about in general terms? \dec{Survey}
\begin{enumerate}
\item Patterns that \emph{match} and any \emph{exceptions}. 
\begin{itemize}
\item Current computer programs are able to identify known patterns
  and ``close matches'' in data sets from certain domains, like music
  \cite{meredith2002algorithms}.  Identifying known patterns is a
  special case of the more general cocept of \emph{pattern mining}
  \cite{bergeron2007representation}.  In particular, the ability to
  extract \emph{new} higher order patterns that describe exceptions is
  an example of ``learning from feedback.''  Deep learning and
  evolutionary models increasingly use this sort of idea to facilitate
  strategic discovery \cite{samothrakis2011approximating}.  Similar
  ideas are considered in business application under the heading
  ``process mining'' \cite{van2011process}
\end{itemize}
\item Progress relative to explicit or adduced \emph{exploration}
  (knowledge and accuracy) or \emph{exploitation} (directional,
  task-based) goals.
\item \emph{Quantity}, \emph{Variety}, and \emph{Order} of produced
  objects or behaviours
\item New relationships among produced objects and behaviours drawing
  on a common field of reference
\end{enumerate}
\item[] How can feedback be understood and used? \dec{Survey}
\begin{enumerate}
\item Update knowledge base with new facts (accept statements,
  possibly with provenance)
\item Note similarity to \emph{iterative development}
\item Reflection: describe relationships between produced objects and
  behaviours and feedback
\item Reflection: Higher-order patterns e.g.~new patterns that
  describe the identified exceptions
\end{enumerate}
\end{enumerate}
