\section{Related work}\label{sec:ww-related}

% \subsection{Feedback in general terms}

In considering the potential and contribution of the Writers Workshop model outlined in Section \ref{sec:writers-workshop}, we now consider how the [*** INTRO AS TO WHY OUR STUFF WILL SAVE THE WORLD AT LEAST 3 percent ***] [TURN THIS INTO AN INTRODUCTORY SECTION 1-2 PARAGRAPHS ON HOW FEEDBACK IS IMPORTANT AND THE DIFFERENT FORMS IN WHICH IT HAS COME UP IN AI BEFORE - THEN NARROW DOWN TO LOOK MORE SPECIFICALLY AT MATTERS RELEVANT TO THE WW MODEL]

Feedback has long been a central concept in AI-related fields such as cybernetics \cite{ashby1952,seth2015cybernetic}. QUOTE

Feedback about feedback (\&c for higher orders) is understood to be
relevant to thinking about \emph{learning} and \emph{communication}
\cite{geertz1973interpretation}.

% These concepts are biologically relevant: for example, immunity (and
%autoimmunity) are examples of biological feedback systems.  They are
%also psychologically relevant: According to Seth (2013) [***CITE ***], emotional
%states are perception of physiological state but ``cognitive context''
%matters.  *** CANT SEE HOW THIS FITS IN

%Feedback and social interaction: Cf.~the Human Interaction Ontology. REFERENCED IN DISCUSSIONS

\subsection{Feedback in computational creativity} \label{ref:related-computational-creativity}

\emph{Creativity} is often envisaged as cyclical process
(e.g.~Dickie's \citeyear{dickie1984art} art circle, Pease and Colton's
\citeyear{pease2011computational} Iterative
Development-Expression-Appreciation model).  There are opportunities
for embedded feedback at each step, and the process itself is ``akin
to'' a feedback loop.  However, despite these strong intimations of
the central importance of feedback in the creative process, our sense
is that feedback has not been given a central place in research on
computational creativity.  In particular, current systems in
computational creativity, almost as a rule, do \emph{not} consume or
evaluate the work of other systems.\footnote{An exception that proves
  the rule is Mike Cook's {\sf AppreciationBot}
  (\url{https://twitter.com/AppreciationBot}), which is a reactive
  automaton that ``appreciates'' tweets from {\sf MuseumBot}.}

\citeA{gervas2014reading} theorise a creative cycle of narrative
development as involving a Composer and an Interpreter, in such a way
that the Composer has internalised the interpretation functionality.
Individual creativity is not the poor relation of social creativity,
but its mirror image.  Nevertheless, even when computer models
explicitly involve multiple agents and simulate social creativity
\cite<like>{saunders2001digital}, they rarely make the jump to involve
multiple systems.  The ``air gap'' between computationally creative
systems is very different from the historical situation in human
creativity, in which different creators and indeed different cultural
domains interact vigourously.

\subsection{Feedback in computational serendipity} \label{ref:related-computational-serendipity}

The term computational serendipity is rather new, but its foundations
are well established.  \citeA{grace2014using} studies \emph{surprise}
in computing.  This latter work seeks to ``adopt methods from the
field of computational creativity [$\ldots$] to the generation of
scientific hypotheses.''  This is an example of an effort focused on
computational \emph{invention}. (Say more here, cite Pease paper.)

Current computer programs are able to identify known patterns and
``close matches'' in data sets from certain domains, like music
\cite{meredith2002algorithms}.  Identifying known patterns is a
special case of the more general concept of \emph{pattern mining}
\cite{bergeron2007representation}.  In particular, the ability to
extract \emph{new} higher order patterns that describe exceptions is
an example of ``learning from feedback.''  Deep learning and
evolutionary models increasingly use this sort of idea to facilitate
strategic discovery \cite{samothrakis2011approximating}.  Similar
ideas are considered in business application under the heading
``process mining'' \cite{van2011process}.

\subsection{Communications and feedback}

Gordon Pask's conversation theory, reviewed in
\cite{conversation-theory-review,boyd2004conversation}, goes
considerably beyond the simple process language of the workshop,
although there are structural parallels.  In a basic Pask-style
learning conversation \cite[p. 190]{boyd2004conversation}: 

\begin{center}
\begin{fminipage}{.8\columnwidth}
\begin{minipage}{1\textwidth}
\begin{enumerate}[itemsep=0pt,rightmargin=10pt]
\item Conversational participants are carrying
out some actions and observations;
\item Naming and recording what action is being done;
\item Asking and explaining why it works the way
it does;
\item Carrying out higher-order methodological discussion; and, 
\item Trying to figure out why unexpected results occured
\end{enumerate}
\end{minipage}
\end{fminipage}
\end{center}

Variations to the underlying system, protocol, and the schedule of
events should be considered depending on the needs and interests of
participants, and several variants can be tried.  On a pragmatic
basis, if the Workshop proved quite useful to participants, it could
be revised to run monthly, weekly, or continuously.\footnote{For a
  comparison case in computer Go, see
  \url{http://cgos.computergo.org/}.}

%\subsection{Our previous work}

In earlier work \cite{corneli15cc,corneli15iccc}, we used the idea of
dialogue in a Writers Workshop framework to sketch a ``theory of
poetics rooted in the making of boundary-crossing objects and
processes'' and described (at a schematic level) ``a system that can
(sometimes) make `highly serendipitous' creative advances in computer
poetry'' while ``drawing attention to theoretical questions related to
program design in an autonomous programming context.''




