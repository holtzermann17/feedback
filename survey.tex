\section{Related work}\label{sec:ww-model}

\subsection{Our previous work}

In earlier work \cite{corneli15iccc,corneli15cc}, we sketched a
``theory of poetics rooted in the making of boundary-crossing objects
and processes'' and ``described a system that can (sometimes) make
`highly serendipitous' creative advances in computer poetry'' while
``drawing attention to theoretical questions related to program design
in an autonomous programming context.''

What is most missing in these earlier papers is an implementation.
Although we do not make up for this in the current paper, we sketch
concrete implementation plan.  In order to bring in other
participants, we need a neutral environment that is not hard to
develop for: the \Fw\ system described in Section
\ref{sec:implementation} offers one such possibility.  With this
system, the basic operating logic of the Workshop could be spelled out
as a flowchart, and contributing systems could use flowcharts as the
basic medium for sharing their presentations, feedback, and questions.
Developing around a process language of this sort partially obviates
the need for participating systems to have strong natural language
processing capabilities.

\subsection{Communications and feedback}

Gordon Pask's conversation theory, reviewed in
\cite{conversation-theory-review,boyd2004conversation}, goes
considerably beyond the simple process language of the workshop,
although there are structural parallels.  In a basic Pask-style
learning conversation \cite[p. 190]{boyd2004conversation}: 

\begin{center}
\begin{fminipage}{.8\columnwidth}
\begin{minipage}{1\textwidth}
\begin{enumerate}[itemsep=0pt,rightmargin=10pt]
\item Conversational participants are carrying
out some actions and observations;
\item Naming and recording what action is being done;
\item Asking and explaining why it works the way
it does;
\item Carrying out higher-order methodological discussion; and, 
\item Trying to figure out why unexpected results occured
\end{enumerate}
\end{minipage}
\end{fminipage}
\end{center}


Naturally, variations to the underlying system, protocol, and the
schedule of events should be considered depending on the needs and
interests of participants, and several variants can be tried.  On a
pragmatic basis, if the Workshop proved quite useful to participants,
it could be revised to run monthly, weekly, or
continuously.\footnote{For a comparison case in computer Go, see
  \url{http://cgos.computergo.org/}.}

\begin{enumerate}[start=2]
\item \textbf{Here we refine the idea and turn it into a general model
  that for incorporating feedback within computational models of the
  creative process.}
\item[] Intuitive examples of application areas
\begin{enumerate}
\item learning/training
%% Supervised learning,
%% Reinforcement learning (Temporal Difference Learning)
%% Evolution
\item guidance/control
\end{enumerate}
\item[] What can feedback be about in general terms? \dec{Survey}
\begin{enumerate}
\item Patterns that \emph{match} and any \emph{exceptions}. 
\begin{itemize}
\item Current computer programs are able to identify known patterns
  and ``close matches'' in data sets from certain domains, like music
  \cite{meredith2002algorithms}.  Identifying known patterns is a
  special case of the more general concept of \emph{pattern mining}
  \cite{bergeron2007representation}.  In particular, the ability to
  extract \emph{new} higher order patterns that describe exceptions is
  an example of ``learning from feedback.''  Deep learning and
  evolutionary models increasingly use this sort of idea to facilitate
  strategic discovery \cite{samothrakis2011approximating}.  Similar
  ideas are considered in business application under the heading
  ``process mining'' \cite{van2011process}
\end{itemize}
\item Progress relative to explicit or adduced \emph{exploration}
  (knowledge and accuracy) or \emph{exploitation} (directional,
  task-based) goals.
\item \emph{Quantity}, \emph{Variety}, and \emph{Order} of produced
  objects or behaviours
\item New relationships among produced objects and behaviours drawing
  on a common field of reference
\end{enumerate}
\item[] How can feedback be understood and used? \dec{Survey}
\begin{enumerate}
\item Update knowledge base with new facts (accept statements,
  possibly with provenance)
\item Note similarity to \emph{iterative development}
\item Reflection: describe relationships between produced objects and
  behaviours and feedback
\item Reflection: Higher-order patterns e.g.~new patterns that
  describe the identified exceptions
\end{enumerate}
\end{enumerate}

\begin{enumerate}[start=4]
\item \textbf{Discussion of how this would work more generally in
  computational creativity and perhaps in AI more generally with an
  eye toward producing effects like ``serendipity'' and
  ``emergence.''}
\begin{enumerate}
\item Section \item Feedback is the fundamental concept in \emph{cybernetics}.  \dec{Definition?}
\item ``Artificial and natural intelligence are both search'' -- you learn a few concepts the hard way and then use them to understand things you only have one shot at (Metaphors we live by).  Neo-diffusionist hypothesis: culture is like any other feature, if it helps you, you're more likely to keep it.  (Kashima).  Deacon's ``Theory of semantics''.  Human semantics can be replicated by statistical learning on large corpora.  Finch 1993, Bilovich and Bryson 2008.  The 75 second most frequent words give a good match to human semantics.  McDonald and Lowe, check the cosines.  Implicit University of Bath, Macfarlane.  Bryson 2008, Theory of Semantics.  ``What are the things that need to be preserved and what are the things that we can mutate?''  Primates not only remember how people treat themselves, they also remember how they treat each other (Second Order representation).  Evolvability is something that AI reseachers should be getting their hands on.  Dual replicators: culture and biology both evolve.  Within the genome, there are hierarchical representations and genes to flag ``zones of innovation'' (Richard Watson, machine learning, genetics, and evolvability)  Cross-over lets you stay close to the fitness peak (Yifei Wang).  \v{C}a\v{c}e and Bryson 2007, agent based modelling of communication costs, in Emergence and Evolution of Linguistic Communication.  ``What is transmitted is the replicator, but the unit of selection is the vehicle/interactor.''  Simpson's paradox.  Daniel Taylor, Evolution of the Social Contract -- everything is semi-private, disseration.  Lots of ways to create modular systems.  Jekaterina Novikova, transparently synthetic emotions for collaboration.  Swen Gaudl, learning from observation of human players and then coming up with better versions of the game using genetic programming.  London Futurists.  Human language and our values come out of our experiences and our \emph{sociality}.  E.g. punishing someone for being antisocial is contributing to your own good.  ``Chimps are moral patients.''
\begin{itemize}
\item (Clark 2013) quotes Ashby.  Homeostatics\ldots Stuff from Anil's talk.
  Wiener 1964 cites Ashby.  Conant and Ashby: ``Every good regulator
  of a system must be a model of that system.''  This leads to the
  ``free energy principle.''  Keep the organism within states in which
  it will receive the kinds of input it expects.
\item E.g.~emotional states are perception of physiological state but ``cognitive context'' matters.  (Seth 2013)
\end{itemize}
\item Feedback about feedback (\&c for higher orders) is relevant to thinking about \emph{learning} and \emph{communication}
\item \emph{Creativity} is often envisaged as cyclical process (e.g.~Dickie's
  art circle, Colton et al.~Iterative
  Development Expression Appreciation).  There are opportunities for
  embedded feedback at each step, and the process itself is ``akin
  to'' a feedback loop.
\begin{itemize}
\item A very intuitive point: customer feedback is a big thing, and increasingly present in online commerce and the social web (think e.g.~of Yelp).  The particular challenge here is to have the resources in order to effectively respond to the feedback.
\item Immunity (and autoimmunity) are examples of biological feedback systems.  Cf.~the Human Interaction Ontology.
\item Another philosophical point (maybe we can bring it in earlier)
  is that in the Writers Workshop model, feedback is somehow an
  economic ``externality.''  That is, the author gets the feedback
  ``for free.''  Other applications of the idea would be similar.
  This kicks off a number of related questions.
\item The first question is about ``how integrated'' the systems
  should be.  Are we willing to call an action taken by the system
  itself `feedback'?  Probably not: for that appelation we would
  require interaction with the world, causally determined but in some
  way stochastic effects.
\item This is related to Bergson's discussion of \emph{reflex} and
  (especially) Coase's discussion of firms.  In a more AI setting,
  this relates to the idea of \emph{sensors} and \emph{effectors}.  In
  dialogue models there are ideas of ``deontic scoreboards''
  (Brandom, Walton).
\item Different traditions in cognitive science: predictive processing
  e.g.~von Helmholtz.  The brain has no direct access.  There are
  uncertain signals about actual things in the world; so the brain
  needs to use inference.  (Clark 2013 BBS) The beholder's share.
  This maps onto ideas of predictive processing in cortical networks.
  Top down convey predictions, bottom up convey prediction errors
  (French 2009 Trends in Cognitive Science).  In the end, there's a
  sensible explanation.  We usually think that bottom up conveys the
  signal; but this outside-in pathway is built on feedback.  How
  strongly should we weight prediction errors versus predictions?
%% Sellars: reliable differential capacities to respond to environmental stimuli
\end{itemize}
\end{enumerate}
\end{enumerate}
